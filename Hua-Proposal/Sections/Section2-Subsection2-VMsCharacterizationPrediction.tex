\subsection{VM Characterization and Performance Prediction}


There have been numerous previous studies of performance prediction for optimization purpose~\cite{Yang2005} ~\cite{Matsunaga2010}. A lot of these researches are built based on data analysis and modeling techniques, depending on in-depth understanding of the application. Most of these studies come up with models of high accuracy, but these models are usually application specific ~\cite{Shan2008}. For a different application, researchers have to develop a new model and evaluate it in a specific environment. But if we can categorize any application into a certain class by clustering analysis upon same set of characteristics, user can optimize performance for a specific application and come up with some results (e.g. fix numbers of some characteristics of VM), which can be reused for scheduling ~\cite{OguraM10}and allocating VMs for some other different applications within the same class. However, there is not such a complete products can categorize applications in cloud and be utilized by different researchers.

Due to the way of categorization of applications can be very different depending on the angel of the observer. ~\cite{Derrick2014} categorize the application based on the job’s intrinsic structure (i.e., how many tasks per job and how to connect them), they split the applications into four execution types (or application types), single-task application, sequential-task application, batch-task application and mix-mode application. While from the perspective of other researches, one application cloud be devided into web server, transactional database, load balancer,etc.

There are a lot of classification algorithms for predicting the class of an object by analyzing historical data different algorithms: k-nearest neighbor (k-nn) algorithm, Linear Regression (LR) algorithm, Decision table/tree (DT) algorithm, Support Vector Machine (SVM) algorithm, Neural Networks (NN). For example, ~\cite{Matsunaga2010} predicted the time and resources consumed by applications using a Decision table/tree (DT) algorithm. But each algorithm will have the different performace and accurcy. In ~\cite{Samuel2013}, Samuel evaluated three algorithms, including SVM, LR, NN, to predicit response time of the application, where he claimed the SVM has the best accuracy by comparing the error rates. So we also need to evaluate the result of different algorithms and choose the best fit to obtain good accuracy.