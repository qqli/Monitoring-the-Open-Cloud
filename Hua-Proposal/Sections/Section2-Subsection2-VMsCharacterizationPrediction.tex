\subsection{VM Characterization and Performance Prediction}
There have been numerous previous studies of performance prediction for optimization purpose~\cite{Yang2005} ~\cite{Matsunaga2010}. A lot of these researches are built based on data analysis and modeling techniques, depending on in-depth understanding of the application. Most of these studies come up with models of high accuracy, but these models are usually application specific ~\cite{Shan2008}. For a different application, researchers have to develop a new model and evaluate it in a specific environment. But if we can categorize any application into a certain class by clustering analysis upon same set of characteristics, user can optimize performance for a specific application and come up with some results (e.g. fix numbers of some characteristics of VM), which can be reused for scheduling and allocating VMs for some other different applications within the same class. However, there is not such a tools can categorize applications in cloud and it requires the cooperation between cloud providers, research communities and cloud users to accomplish this goal.

Recent year, many papers have been studying individual performance modeling techniques for parallel applications and systems optimization ~\cite{Brian98}~\cite{Yang2005}. All classes of individual techniques have important strengths and weaknesses. Abstract, analytical models provide significant insights into application performance and are usually extremely fast, but lack the ability to capture detailed performance utilization, and most such models must be constructed manually, which limits their accessibility to users. In contrary, program-driven simulation techniques can capture detailed performance utilization at all layers but, can be extremely expensive for large-scale parallel programs and large scale infrastructure, not only in terms of simulation time but especially in their computational resources ~\cite{Huaxia1999} ~\cite{Michael2015}. These methodologies are not perfectly applicable to modern cloud, unless the technical challenges are solved.

Another issue is data collected form cross cloud platform needs to be normalized before actual analysis. Specifically, users of high-performance computing (HPC) platforms tend to have access to more and more geographically distributed computational resources. Unfortunately, both the resources and the applications in today’s distributed computing environment are highly heterogeneous and of great complexity ~\cite{Yang2005}. This makes it difficult to measure the resource usage of a specific application on a wide range of execution platforms. Instead of translating the data set after gathering from different infrastructures, collecting data from a cross environment and storing the data in a meaningful and amenable manner will be more feasible and efficient, because it can avoid extra efforts and research resources in the normalization. MOC is providing such a heterogeneous environments, having different vendors ranging from lenovo, Cisco, Brocade, Intel, where will enable us solve this problem in a real environment.