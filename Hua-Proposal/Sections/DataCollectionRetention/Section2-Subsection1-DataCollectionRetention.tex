\subsection{Data Collection and Retention of Physical Infrastructure}

% There are various data collection and monitoring tools focusing on different layers of the cloud. Nagios~\cite{nagios,nagiosdoc} is a widely-used enterprise-level open source monitoring tool that can monitor cloud's physical infrastructure and some of its services. Sandoval et al.~\cite{sandoval2012evaluation} analyze several already available monitoring tools, such as Nagios, HypericHQ, Lattice, Zenoss, and indicate that Nagios is the best choice. Sensu~\cite{sensu} is designed as a publish/subscribe based monitoring tool that can deal with some of the problems that Nagios cannot solve. In Nagios, there is a configuration file that needs to be modified and we need to restart Nagios whenever we want to make any configuration modification. However, sensu-clients, which run on remote hosts, can subscribe to server defined checks and can avoid restarting the whole monitoring tool. It also uses RabbitMQ, which is a message-oriented middleware that implements Advanced Message Queueing Protocol (AMQP). Sensu is a monitoring tool that is scalable, extensible and elastic~\cite{Aceto2013}. 
% We plan to evaluate Sensu and Nagios as data collection tools. 

% Since MOC infrastructure relies heavily on OpenStack~\cite{sefraoui2012openstack}, we also plan to investigate monitoring solutions available in OpenStack. Ceilometer~\cite{ceilometer} is OpenStack’s metering solution. It can consume notifications from OpenStack message queue and can poll OpenStack services to get their data. Ceilometer stores its data in MongoDB by default. StackTach~\cite{stacktach} is not directly an OpenStack service, but still an alternative to Ceilometer for metering and billing. It can also consume notifications from OpenStack message queue and store the data to a database. We will also evaluate these two cloud middleware layer monitoring tools in our future studies.

Based on the essentially different goals we are focused on, we will be using a combination of layer specific monitoring tools for the physical infrastructure rather than using the full cloud monitoring tools mentioned above. Specific monitoring tools that we are considering include Nagios~\cite{nagios,nagiosdoc}, HypericHQ~\cite{hyperichq}, Lattice~\cite{clayman2010monitoring}, Zenoss~\cite{zenoss}, Zabbix~\cite{zabbix}, and Sensu~\cite{sandoval2012evaluation,Aceto2013}. From our initial analysis, it seems clear that no one tool has the necessary combination of properties necessary for the project, and we expect to be using a combination of different tools.