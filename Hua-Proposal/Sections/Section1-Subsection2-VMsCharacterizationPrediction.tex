\subsection{VM Characterization}
VM characterization is the task of identifying why a VM is used for. For example, is the VM running a Web Server, a caching server, a database application, etc… VM characterization is important because it enables performance optimizations such as, suggesting best VM configurations for different VM classes and making intelligent VM placement decisions. 

The research challenges in this phase can be listed as:
\begin{itemize}
\item Identifying possible VM classes: coming up with a complete VM functionality classification table. There are various types of applications running in the cloud. A VM can be used as database, web server, load balancer, or a system for multiple applications. Our project goal is to characterizing the VM based on its application. So a clear, complete and fine-grained list of VM functionality is needed so that all VMs can be identified as a certain type.

\item Identifying meaningful/useful features, doing feature selection among many data sources and features: There are diversed performance metrics including virtual resources utilization, physcial resournce utilization, power consumptions, etc, some of which may not be releavant enough to the application type for prediction. Additionally, taking account into too many characterics of a VM will bring the performance issue to classification program . Specially, it sometime takes a couple of hours to finish the the computation of classifier calculation. Hence, the selection of features of is must to ensure the efficiency. However, the more features we have, the more difficult it is to accomplish it. Therefore, it's a big challenge to select the best set of features from all the performance metrics our platform can collect.

\item Identifying best classification algorithm among the multiple possible classification algorithms: There are a lot of classification algorithms for predicting the class of an object by analyzing historical data with different algorithms: k-nearest neighbor (k-nn) algorithm, Linear Regression (LR) algorithm, Decision table/tree (DT) algorithm, Support Vector Machine (SVM) algorithm, Neural Networks (NN). However, each algorithm has it's weakness and strength, and different use scenarios. But there is no such a certain algorithm must be the best fit for our application prediction, which we need to solve.

\item Performing scalable/real-time classification, if possible with a small amount of data after a short amount of execution time, to enable early optimizations: most of the time consumed by the prediction process is the computation of the classification model, which is usually calculation upona  large scale of input data. It can take more than a couple hours to finish the computations [], which is not acceptable if we need the output for an prompt resource scheduling. A scalable classification method must be proposed for user experiences.
\end{itemize}