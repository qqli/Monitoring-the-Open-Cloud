\subsection{VM Characterization}
VM characterization is the task of identifying what a VM is used for. For example, is the VM running a Web Server, a caching server, a database application, etc. VM characterization is important because it enables performance optimizations such as, suggesting best VM configurations for different VM classes and making intelligent VM placement decisions. 

The research challenges in this phase include:
\begin{itemize}
\item Identifying VM classes: coming up with a complete VM functionality classification table. Fundamental to our goal of characterizing a VM is coming up with a useful classification scheme such as: database, web server, load balancer, HPC application, NFV, memcached, etc.  

\item Identifying meaningful/useful features for classification: There is a wide diversity of metrics including VM name \& description, virtual resources use, physical resource use, power consumptions, I/O operations, etc.  Identifying the most relevant information will be critical for performance; it can takes hours to finish the computation of classifier calculation. 

\item Selecting a classification algorithm: There are a variety of classification algorithms including: k-nearest neighbor (k-NN) algorithm, Linear Regression (LR) algorithm, Decision tree (DT) algorithm, Support Vector Machine (SVM) algorithm, Neural Networks (NN). We will need to evaluate which algorithm best fits our needs from a performance and accuracy perspective. 

\item Real-time classification: While off-line classification would be valuable, if we can classify the VMs as they are running it would be enormously more valuable, allowing resource management decisions to be made on the fly.  A stretch research goal of our project is to try to develop a scalable classification scheme that can work with limited information. 

\end{itemize}
