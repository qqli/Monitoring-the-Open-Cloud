% ************************************************************************
% ****************** Prior Work ****************************************


\section{Prior Work}
\label{sec:PriorWork}

Monitoring is a core function that cloud should support to meter the complex infrastructure, ranging from physical resources to virtualized resources. The monitoring data has been used for capacity and resource planning and management, data center management, SLA management, billing, troubleshooting and performance management~\cite{Aceto2013}. A lot of works has been done to improve granularity, accuracy, scalability, adaptability and elasticity of monitoring solutions in cloud. Rich monitoring solutions [13] [14] [15] have been proposed in the recent years aimed to deal with all above issues. The mature monitoring solutions in the cloud prove the possibility of collecting resources measurements and user behaviors in different layers in the cloud. I have started investigating Sensu, which is a plugin-architected, scalable, adaptive monitoring application for cloud infrastructure.
There have been some literatures mentioned how to collect data for machine learning research. In [6], the authors set up different sets of computers running HPC applications with different numbers of CPU cores, obtained different performance by manual human changes of the number of cores. This method is apparently not scalable and very specific for this research. In [16], the author proposed an idea about optimizing specific application performance by analyzing data of partial execution of the application, which is actually weakening of the power of historical data. [17] realize the major problems for most researches that it is both labor and resource intensive, and does not provide any assurance of the quality of results. So they proposed a simulation tool for predicting application performance in different simulated cloud structures. However, this simulation tools fail providing a real resources usage as the true cloud environment. These works convince us it’s necessary to provide researches community access to information of an actual cloud environment.
OpenStack is an open source project supported by a large community of developers and companies that allows creating and managing large-scale Cloud IaaS deployment. OpenStack provides Compute service for controlling the VM lifecycle and configuration in specific node, and Network service to maintain and manage the virtual networks. Apart from other services, OpenStack provides a metering service called Ceilometer to reliably collect measurements of the utilization of the physical and virtual resources from user perspective, persist these data for subsequent retrieval and analysis [11]. Because of the flexibility of OpenStack, some researches about cloud efficiency have been done on it. More importantly, its metering service provides rich measurement data, so it is the best cloud environment for my project [12].
