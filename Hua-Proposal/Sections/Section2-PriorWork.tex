% ************************************************************************
% ****************** Prior Work ****************************************


\section{Prior Work}
\label{sec:PriorWork}

Monitoring is a core function of cloud keeps track of the usage and status of infrastructures, ranging from physical resources to virtualized resources. The monitoring data has been used for capacity and resource planning and management, data center management, SLA management, billing, troubleshooting and performance management ~\cite{Aceto2013}. A lot of works has been done to improve granularity, accuracy, scalability, adaptability and elasticity of monitoring solutions in cloud. Rich monitoring solutions ~\cite{Alcaraz2015} ~\cite{Jesus2013} ~\cite{Javier2013} have been proposed in the recent years aimed to deal with all above issues. The mature monitoring solutions in the cloud prove the possibility of collecting resources measurements and user behaviors in different layers in the cloud. I have started investigating Sensu, which is a plugin-architected, scalable, adaptive monitoring application for cloud infrastructure. Nagios will be investigated since it?s highly recommended for complicated cloud infrastructures monitoring in many literatures [14] [15].
OpenStack is an open source project supported by a large community of developers and companies that allows creating and managing large-scale Cloud IaaS deployment. OpenStack provides Compute service for controlling the VM lifecycle and configuration in specific node, and Network service to maintain and manage the virtual networks. Apart from other services, OpenStack provides a metering service called Ceilometer to reliably collect measurements of the utilization of the physical and virtual resources from user perspective, persist these data for subsequent retrieval and analysis [11]. Because of the flexibility of OpenStack, some researches about cloud efficiency have been done on it. More importantly, its metering service provides rich measurement data, so it is the best cloud environment for my project [12].
In literature, machine learning and data mining have been applied for some cloud researches. [21] utilized of data mining techniques to network traffic analysis performed on traffic, packet payloads, traffic metrics, or some statistical features computed on traffic flows. [7] proposed a research on use of different machine learning algorithms to predict the CPU, memory, network consumption and compare the accuracy of them. Additionally, [16]  proposed parallel performance prediction built upon machine learning modeling to help application owners in their research planning and daily use of diverse computing resources. 
