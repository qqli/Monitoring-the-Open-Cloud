\section{Proposed Approach}
\label{sec:ProposedApproach}
First I will focus on data collection for potential performance analysis. First step, I will read more papers about how people want to apply machine learning techniques for that topic. Next, I will investigate tools for data monitoring and metering service for OpenStack for actual data collection in different layers. And I will set up these candidates in MOC and choose one solution for long-term use.
Then I will propose a data storage solution which includes designing a proper data schema and investigating the appropriate backend storage and integrating the backend into the cloud. One time series database and MongoDB has been investigated for this purpose. Some more questions about the query performance, scalability are still waiting to answer before making decision. 
To keep track the user behavior on application level, I will enhance OpenStack user interface called Horizon to record what kind of application has been runned in the VMs by user historically. The application type can be Database, Hadoop, BigTop, etc. This aims to correlate the application a user run with the resource utilization. Since Horizon is implemented with Django, a python web framework, so I will customize Horizon in MOC with this python framework. Detail of implementation is still waiting for design.
And later on, as practice, I will implement a categorization program to group the VMs by historical application usage and another program to predict the application a user is running on a new VMs. I will use K-means algorithm for categorization, which is usually the first option to cluster data in a centroid model. The prediction method can be K-nearest neighbors algorithm, which is a good fit for this scenario.
