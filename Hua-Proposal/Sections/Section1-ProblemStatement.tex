\section{Problem Statement}
\label{sec:ProblemStatement}

\subsection{Data Collection and Retention}

Cloud environments are enourmously complicated systems that are composed of seven very different layers (facility, network, hardware, OS, middleware, application, and the user~\cite{spring2011monitoring}) with complex interactions between the components and players at these different layers.  Our fundamental goal for this phase of the project is to collect information from all these different layers and retain it so that tools can be developed that can correlate information from all the different sources.  For example, a performance tool may need to connect a VM launched by a user to information from cloud middleware about what node the VM was launched on, to information from a hypervisor about what the demands of the VM were. 

% Collection of information about the cloud is done by monitoring tools that 
% There are no tools that monitor all these different layers
% with tools grouped into high-level related to cloud virtual system which is composed of middleware, application, and user layer ~\cite{Aceto2013}. Low-level monitoring is related to cloud’s physical system which is composed of the other four layers.

The research challenges for this phase are 1) how to collect operational data of the cloud from different sources in a scalable way without impacting the performance of the components being monitored, and 2)  how to retain and update the data in a scalable manner, to support multiple types of queries, and to optimize the retention for data analysis.

For data collection, while there exists no tools that collect informaton from all the different layers of the cloud, there are a wide variety of tools that are specific to different layers.  Our hypothesis for this phase is that we can use a combination of existing monitoring tools for data collection, so that we do not have to build new monitoring tools.  We will, however, need to evaluate the different alternatives from both a capability and performance perspective.  
For data retention, there are a variety of different data bases and architectures to evaluate from a scalability and performance perspective.  We will also need to consider if all the data should be incorporated into a single data base, or of information should be partitioned across a data base we define and monitoring tool specific databases. 

%In order to expose the cloud performance characteristics and utilization data to users as well as researchers, we will build a %monitoring platform that can collect and retain the performance and utilization of the cloud. The challenges for this phase %are 1) how to collect operational data of the cloud from different sources in a scalable way without impacting the performance %of the components being monitored, and 2) how to retain the data in a scalable manner and to optimize the retention for data %analysis.
%
%Cloud environment is a complicated system, and it can be modeled in seven layers: facility, network, hardware, OS, middleware, %application, and the user~\cite{spring2011monitoring}. Monitoring of these seven layers can be grouped to high-level and low-%level monitoring. 
%High-level monitoring is related to virtual cloud resources, which are composed of middleware, application, and user layer~\%cite{Aceto2013}. 
%Low-level monitoring is related to cloud’s physical infrastructure, which is composed of the other four layers. 
%
%There exists multiple successful monitoring tool options for each layer of the cloud. For data collection and retention phase %we plan to use a subset of these tools. However, as there are many options, we need to evaluate and investigate a huge set of %combinations to obtain the performance and utilization of the cloud in an effective and efficient manner. 
%
%We plan to investigate cloud data collection tools such as Sensu, Nagios, Zabbix, Ceilometer, Stacktach, and Logstash. Our %main concerns in data collection is a) having a wide range of data collection capabilities, b) being scalable, c) having a low %impact on the monitored system.  The data that we initially plan to monitor are utilization of the physical resources (e.g. %CPU, memory, disk, network, I/O bandwidth, etc...) and cloud services (e.g. authentication, cloud networking, VM image hosting %services, etc...), power consumption of components of the cloud (e.g. CPU/memory voltage, temperatures of various node %components, etc...), utilization and metering of virtual resources (e.g. how many VM hours per user, how much I/O, VM to %physical node mappings, etc…), logs from different cloud layers (e.g. OS level logs, hypervisor level logs, cloud management %service logs, etc...).   
%
%Similarly, there are multiple database options for retaining the data collected from the cloud. The database solutions we will %adopt has to be scalable, easy to update, amenable to supporting multiple types of queries. We plan to evaluate options such %as InfluxDB, MongoDB, MySQL, etc… Some data collection tools come with a database, while other monitoring tools does not %retain the data they collected. Whether to store all the data collected in one large database or store it in separate %databases as provided by some of the monitoring tools is an issue we will investigate.
%
\subsection{Cloud Monitoring Data Analysis}


There have been numerous previous studies of performance prediction for optimization purpose~\cite{Yang2005} ~\cite{Matsunaga2010} . A lot of these researches are built based on data analysis and mondeling techniques, depending on in-depth understanding of the application. Most of these studies come up with models of high accuracy, but these models are usually application specific ~\cite{Shan2008}. For a different application, researchers have to develop a new model and evaluate it in a specific environment. But if we can categorize any application into a certain class by clustering analysis upon same set of characteristics, user can optimize perfromance for a specific application and come up with some results (e.g. fix numbers of some characteristics of VM), which can be reused for scheduling and allocating VMs for some other different applications within the same class. However, there is not such a tools can categorize applications in cloud and it requires the cooperation between cloud providers, researche communities and cloud users to accomplish this goal.

Recent year, many literatures have been studying individual performance modeling techniques for parallel applications and systems optimization ~\cite{Brian98}~\cite{Yang2005}. All classes of individual techniques have important strengths and weaknesses. Abstract, analytical models provide significant insights into application performance and are usually extremely fast, but lack the ability to capture detailed performance utilization, and most such models must be constructed manually, which limits their accessibility to users. In contary, program-driven simulation techniques can capture detailed performance utilization at all layers but, can be extremely expensive for large-scale parallel programs and large scale infrastructure, not only in terms of simulation time but especially in their computational resources ~\cite{Huaxia1999} ~\cite{Michael2015}. These methodologies are not perfectly applicable to modern cloud, unless the tehcnical challenges are solved.

Another issue is data collected form cross cloud platform needs to be normalized before actual analysis. Specifically, users of high-performance computing (HPC) platforms tend to have access to more and more geographically disributed computational resources. Unfortunately, both the resources and the applications in today’s distributed computing environment are highly heterogeneous and of great complexity ~\cite{Yang2005}. This makes it difficult to measure the resource usage of a specific application on a wide range of execution platforms. Instead of translating the data set after gathering from different infrastructures, collecting data from a cross environment and storing the data in a meaningful and amaenable manner will be more feasible and efficient, because it can avoid extra efforts and research resources in the translation. MOC is providing such a heterogeneous environments, having different vendors ranging from lenovo, Cisco, Brocade, Intel, where will enable us sovle this problem in a real environment.



Cloud computing, as a new industry with increasing number of users, has many intrinsic advantages. With clouds, developers with innovative ideas can try out their new services, without investing in huge amounts of hardware [] It is also claimed that cloud computing gives users the illusion of infinite computing resources available on demand and allows users to pay for the use of computing resources on a short-term basis as needed [] However, there are still some challenges for cloud computing, such as performance invisibility, data transfer bottlenecks, scalability, and cloud information are kept away from users for optimizations.
Cloud providers do not provide users with any information about the underlying hardware infrastructure of public clouds, thus promising users the illusion of ideal virtual machines. On the contrary, those virtual machines are far from ideal, performance is hardly visible. Michael Conley et al. [17] state in their paper that they conduct a series of experiments in order to understand the scaling behavior of virtualized cloud network and storage resources. Far from getting a satisfactory result, they spent 50,000 dollars getting a result that is highly that may not be repeated. If researchers in [17] are provided with sufficient operational data of physical system or other experimental results of clouds, less effort and less cost would be spent for similar series of experiments and more techniques can be applied to cloud research, such as machine learning and data mining.
Despite of performance invisibility of clouds, cloud providers also hide network bandwidth information from users. As many applications are bandwidth bound, such as a video streaming application, users may have a job finished much longer in a congested network than in a network that is free. But they are unaware of whether their network is congested or idle, making choices only based on feelings. 
Even though clouds have many challenges and clouds need to be monitored as stated above, most public clouds, which usually run by a single vendor, hide their operational data from users of the clouds. The Open Cloud Exchange (OCX) is envisioned as a public cloud marketplace in which many stakeholders, rather than just a single provider, participate in implementing and operating the cloud [3]. It is also envisioned that cloud?s operational data can be exposed to researchers for their studies. Massachusetts Open Cloud (MOC) is an experiment environment of OCX, many vendors, such as Dell, Lenovo, and Intel, participate in this project. MOC is the first open cloud that users can use, and potential researchers can do research with resourceful information. As MOC intend to expose data to researchers as well as users, cloud monitoring is also a must in MOC. In addition, with vendors providing many services, MOC is a place that can give users more choices, including brands, performance, price, energy efficiency and so on. Those choices are only realizable if we provide a monitoring solution to MOC. 
In addition to the monitoring motivation that is triggered by the opacity of current public clouds, cloud providers must monitor the cloud in order to guarantee SLAs. Service provisioning in the Cloud is based on Service Level Agreements(SLAs), which states the term of the service including the quality of service (QoS), obligations, service pricing, and penalties in case of agreement violations [16]. However, cloud environment can be modeled in seven layers: facility, network, hardware, OS, middleware, application, and the user[15]. Traditional monitoring tools cannot be applied to cloud environments, which stress the need of cloud monitoring.
Cloud monitoring can be divided into high-level monitoring and low-level monitoring[15]. Most of the clouds that provide monitoring services only give high-level monitored information, which is the virtual system of the cloud. Low-level monitoring, which can be considered as the monitoring of clouds? physical system, is often not provided to consumers.
Based on the needs of cloud monitoring and the opacity of clouds that hinders users experience and research capabilities of clouds, we propose that exposing operational data of the cloud in a readable, organized and meaningful way to help users in making decisions and doing scientific researches. However, there are still many challenges in terms of cloud monitoring. First, how to collect operational data of the cloud from different sources in a scalable way, without impacting the performance of the components being monitored. Second, how to retain the data in an organized way and to optimize the retention for data analysis. Third, how to make use of that data in various research projects. Fourth, how to expose the information in a way that is clear to users and also protect the privacy of the cloud and users. 
Qingqing Li and I will cooperate in taking the first steps towards the challenges mentioned above. We will spend two months in developing a data collection layer and a data retaining solution in MOC. In this phase, we will aim to set up low-level monitoring in MOC, come up with a basic set of data, collect and retain the information. Then we will separate our tasks into two directions. For completion of my M.S. project, a tool will be developed to collect and retain information about the applications the user is running on their instances.  I will use machine learning techniques to create a clustering model to categorize the application types with the data we will collect and retain. A regression model will be created to predict the applications a user is running with a set of data in our database.
