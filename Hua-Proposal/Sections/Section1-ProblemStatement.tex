\section{Problem Statement}
\label{sec:ProblemStatement}

Performance is always big concern in large-scale cloud and there have been a lot of researches about it. Some people optimize efficiency by improving some resources utilization according to strategies developed based on instantaneous measurement of the system, for example, make virtual machines placement decision basing on the instantaneous workload of the hypervisor and change the virtual machines configuration on fly. Some people tried to run HPC application in different setting of VMs in AWS to find out the most cost-efficient configuration, where the method itself is very inflexible and expensive than other alternatives [4]. However, the constant dynamic nature and complexity of the cloud have made the conventional optimization methods much more challenging than before. With the development of modern machine learning technology, analysis with machine learning is an increasingly appealing approach for solving this complex problem in cloud [6] [7] [8] [10]. Since, using algorithms that iteratively learn from data, machine learning allows computers to find hidden insights without being explicitly programmed where to look. Facts like growing volumes and varieties of available data, computational processing that is cheaper and more powerful, and affordable data storage. All of these things mean it's possible to quickly and automatically produce models that can analyze bigger, more complex data and deliver faster, more accurate results – even on a very large scale [10]. And these advantages of machine learning make it applicable to cloud performance research. Moreover, this means the kind of problem that machine learning can solve could be extended more than just optimization, for instance, examining user behavior to understand user requirements and expectations is for better customer experience [10].
The increasing interest of machine learning for cloud performance research is visible, but application of machine learning technology is still limited and unconvincing from the cloud user perspective because of several reasons. Firstly, most of these works are done with limited information. This approach requires researcher take the historical performance and historical utilization of computing resources into account. While researchers have started analyzing history measurements of CPU and memory for performance optimization, analysis of network usage, applications usage, power usage or system flavor is still sparse. Because none of current cloud provider is publishing any of these information, it’s very challenging for researcher to obtain various data from real cloud environment. Secondly, data may differ in different clusters. Cloud providers have different mixtures of resources to delivers cloud services. To collect data for analysis, researcher needs to pay a lot of efforts to collect data needed [4]. In different cloud, different approaches will be used to obtain training data and they usually come with intensive labor and resource usage. Thirdly, user behavior in the cloud is lacking of observation. In order to improve the user experiences, while understanding of the user behavior is fundamental, considering the performance delivered to the customer in past is also very essential. Some researcher may ask what applications the users have used on their instances, what the correlation between cloud performance and users applications is. No cloud provider is able to provide a way to help academic researchers retrieve data to answer these problems.
In my M.S. project, I will try to solve these issues by proposed a framework in MOC to provide user and research communities the visibilities and accessibilities to this information. First it will address how to collect various system metrics in the MOC. The next step is proposing a fashion to collect user behavior in application level. A way of data retention will be designed as a part shared with another M.S. project about exposing data to cloud user.
