\subsection{VM Characterization}

Our approach for VM categorization phase of the project involves the following tasks: i) Identifying possible VM classes, ii) Forming a ground truth by modifying the IaaS UI such that users can indicate what kind of application they will run in their VMs during VM creation, iii) Using the data coming from the data collection layer and users building various classification models, evaluating the importance/benefit of various information sources/features, and testing and comparing the accuracy of the classification models.  

The first task is to create a generic categorization directory for VMs. When we are building the category directory, we will explore different focuses such as: 1) interests to users; 2) value for research and 3) tools that could exploit the characterization. To obtain this information, beyond continuing to explore the related research, we plan to discuss any proposed characterization, and collect requirements from industry collaborators of MOC and researchers who have interest in building higher level tools. 

The next task is implementing a mechanism into the UI of OpenStack to collect the ``ground truth'' from users by modifying the UI such that users can indicate what kind of application they will run in their VMs during VM creation. The goal of this is to enable users to add classes of VM's manually, and to construct a ground truth dataset for our prediction models.

The last task is building classification models using the data coming from the data collection layer and the ground truth information coming from MOC users. In this task, we will explore multiple classification algorithms such as kNN, linear/logistic regression, SVM, neural networks, decision trees/random forests, etc. Furthermore, we will try to reduce the set of features used for classification by exploring the contribution of different features and selecting the best set of features. To this end we plan to investigate standard machine learning feature selection techniques such as L1 regularization. We will also rank features based on their correlation with the categorization target. Based on this, we will have an understanding of features to be used in an efficient fashion, driving the next step of performance optimization. For measuring accuracy we will apply ten-fold cross-validation on the dataset to avoid over-fitting. As a stretch goal, we will explore if our solution can be realized in a real time setting, and exposed through the IaaS UI to users for verification.
