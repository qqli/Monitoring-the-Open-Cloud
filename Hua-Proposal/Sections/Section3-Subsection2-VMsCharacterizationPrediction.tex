\subsection{VM Characterization and Performance Prediction}

In order to construct a category of applications before prediction, we will investigate different categorization angles in terms of several aspects: 1) interest of the users; 2) amenability in further cloud researches; 3) feasibility of identifying the class 4) or more. This phase is the requirement analysis at the beginning of the life cycle of software development. To abtain this information, we will studies more papers related and discuss with the industry collaborators of MOC or potential researchers who have interest in this tool. The goal of this step is to create the most meaningful category of applications that will be applicable to user's decision making or optimization research.

Since all modeling techniques will need a set of sufficient history data consists of labels (classes of applications) and training set (performance metrics) to construct the prediction model. Therefore, in this phase we will implement a tool to collect the "ground truth" from users by modifying the IaaS UI such that users can indicate what kind of application they will run in their VMs during VM creation. The goal of this tool is to enable user adding labels of applications manually to construct the training data for our prediction modeling analysis.

The next phase is selecting features within the rich performance metrics collected from different resources via monitoring tools. We have realized the diversity of the performance metrics that can be collected from the cloud infrastructures. To achieve the accuracy of the prediction output and reducing time consumption of the program, we must need to select the features with careful investigation in terms of the correlationship between the application class and computation performance of the algorithm(s).
