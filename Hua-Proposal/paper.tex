\documentclass[11pt,draft,letterpaper,compsoc]{IEEEtran}
\usepackage{enumerate}
\usepackage{color}
\usepackage[dvipdf,dvips]{graphicx}
\usepackage{epstopdf}
\usepackage{subfigure}
\usepackage[cmex10]{amsmath}
\usepackage{algorithm}
\usepackage{algorithmic}
\usepackage{url}

\hyphenation{op-tical net-works semi-conduc-tor}

\begin{document}
\title{BU ECE Dept. MSc Proposal\\ Open Cloud for Cloud Research}

\author{Hua~Li (UXXX) \\ (Advisor(s): Prof.~Orran~Krieger, Dr. Ata Turk)}

\IEEEcompsoctitleabstractindextext{%
\begin{abstract}

With the rapid growth and wide adoption of cloud computing, the complexity of cloud infrastructures is increasing with a lot of challenges, on which the research community and industry are investing a lot of resources. One challenge is provision and continutiy of application performance, which is critical to both cloud privider and cloud usesr, because they mean reliable service, cost efficiency. For application performance optimization, there have been numerous researches about prediction of performance, which helps administrators and users make decision of resources sheduling, planning and allocation. Most of these researches come up with prediction methods with high accuracy. However, these prediction methods are ususally either cloud provider specific or application specific requiring huge extra works to apply them in actual sheduling or planning, moreover, these studies ususally cost the researchers a lot repeative intensive works in collecting data from cloud for experiments. Instead of spending large amount resources and works in answering questions like: what the performance it will be if this kind of application is running on my VMs. If the cloud provider can answer these questions, that will be very valuable to the cloud users and research community. For example, cloud provider can make suggestion to user the configuration of the VMs that is most efficient for specific usage to help usesrs make accurate placemnt descion to achieve high performance. But this is still an illusion which needs the researchers and industry work together to realize it. To achieve this, two of the challenges are (i) performance measurements of cloud infrastructures and (ii) categorization of different applications in cloud. For completion of this M.S. project, we will build a platform to collect and retain measurements data from cloud and a tool to categorize the applications running a VM.

\end{abstract}

\begin{IEEEkeywords}
    Monitoring, Cloud, MOC, Visualization.
\end{IEEEkeywords}}

% make the title area
\maketitle

\IEEEdisplaynotcompsoctitleabstractindextext

\IEEEpeerreviewmaketitle

\section{Problem Statement}
\label{sec:ProblemStatement}

%\subsection{Introduction}

One of the main arguments for utilizing public clouds is it is an environment where developers with innovative ideas can try out their new services without investing in huge amounts of hardware~\cite{Armbrust2009}. However, public cloud providers refrain from exposing the performance characteristics and utilization of the cloud software and hardware components from cloud users and researchers. As a result, developers are often forced to do reverse engineering to discover the characteristics of the cloud and optimize their applications. A good example to this is presented in~\cite{conley2015achieving}, where Conley et al. try to identify the best performing instance type for sorting 100TB of data on Amazon EC2. After evaluating 48 different instance types available, and spending \$50.000 along the way, they identified the best performing instance type. That instance type can sort 100TB of data at a cost of \$300. 

Unfortunately, even the authors of~\cite{conley2015achieving} are not exactly sure that their experiments are repeatable as the utilization of the cloud and the resources they used when they performed these experiments are unknown to them. Our goal in this project is to prevent this kind of performance uncertainty by providing the underlying infrastructure utilization information to users so that they can do intelligent decisions.          

However, there are many valid reasons why cloud providers do not want to provide visibility into the performance and utilization characteristics of their cloud infrastructure. First, they want to keep their solutions private due to commercial concerns. Second, sharing details of the underlying cloud infrastructure may lead to security problems. Third, providing information about a user?s resources may expose information about other users in the cloud and cause privacy issues. 

On the other hand, since we envision to implement our monitoring solution on an open cloud setting (MOC), we are not bound by commercial concerns. Also, we believe that security through obscurity is not the right solution for achieving high level cloud security and providing increased transparency to the cloud infrastructure will foster building of more secure cloud solutions. Finally, one of our goals in this project is to come up with ways of exposing cloud utilization information in a manner that preserves user privacy.

%This project will be composed of two phases. The first phase, Data Collection and Retention, will be a platform building phase that collects and stores raw data from the MOC. It will be a shared effort with another master student.  
%The second phase, Data Transformation and Visualization, will first extend the platform by filtering and transforming the raw data collected from the cloud such that a number of query sets can be efficiently performed over the data. We will then exercise and refine the platform by developing an end-user tool that visualizes both the status of the cloud, and the user’s virtual cloud resources. The goal of this tool will be to provide users the kind of information they have to reverse engineer on existing clouds to optimize their applications.
One of the main arguments for utilizing public clouds is that it is an environment where developers with innovative ideas can try out their new services without investing money in huge amounts of hardware~\cite{Armbrust2009}. However, public cloud providers do not expose the performance characteristics and the utilization of the cloud software and hardware components to cloud users and researchers. As a result, developers are often forced to do reverse engineering to discover the characteristics of the cloud and to optimize their applications. A good example of this is presented in~\cite{conley2015achieving}, where Conley et al. try to identify the best performing instance type for sorting 100TB of data on Amazon EC2. After evaluating 48 different available instance types, and spending \$50,000 along the way, they identified the best performing instance type. That instance type can sort 100TB of data at a cost of \$300. 

Unfortunately, even the authors of~\cite{conley2015achieving} were not confident that their experiments were repeatable. This is because the utilization of the cloud software and hardware and the resources the authors used when they performed these experiments were unknown to them. Our fundamental goal in this project is to expose the information that Conley at al. spent tremendous effort and money trying to discern from outside the cloud. 

There are many valid reasons why cloud providers do not want to provide visibility into the performance and utilization characteristics of their cloud infrastructure. First, they want to keep their solutions private due to commercial concerns. Because we plan to implement our monitoring solution on an open cloud setting, we are not bound by commercial concerns. Second, sharing details of the underlying cloud infrastructure may lead to security problems. Third, providing information about a user's resources may expose information about other users in the cloud and cause privacy issues. 

Since no public cloud vendor has ever exposed the cloud performance characteristic and utilization data to users and researchers, we are facing new with challenges, such as security issues and privacy issues that have not been studied by researchers. In addition, it is challenging to monitor the whole cloud environment, which is composed of seven different layers (facility, network, hardware, OS, middleware, application, and the user~\cite{spring2011monitoring}) with complex interactions between the components and players at these different layers. 

In this project we plan to develop a cloud monitoring and visualization solution that can mediate and display data coming from both the physical and virtual layers of the cloud. We plan to perform these studies on Massachusetts Open Cloud (MOC). As part of this project, we will develop a virtual layer monitoring solution, and then combine the outputs of this solution with the outputs of a physical monitoring solution that is being developed for the MOC by another master student. One goal of this project is to collect and retain information from all these different layers so that tools can be developed that can correlate information from all the different sources. For example, a performance tool may need to connect a VM launched by a user to information from cloud middleware about what node the VM was launched on, to information from a hypervisor about what the demands of the VM were. 

We then propose to develop a data transformation and visualization tool that will expose meaningful data to the MOC tenants and cloud researchers. This tool will first extend the monitoring platform by filtering and transforming the raw data collected from the cloud so that a number of query sets can be efficiently performed over the data. Then we will exercise and refine the platform by developing an end-user tool that visualizes both the status of the cloud, and the user’s virtual cloud resources. The purpose of this tool is to provide users with the information they have to reverse engineer on existing clouds to optimize their applications.

The research challenges for this project are 1) how to make the platform scalable to serve many queries, 2) what kind of queries are necessary to ensure privacy while providing valuable information to users, 3) to what extent should information be exposed to ensure security of the cloud system, and 4) how to relate the data of different cloud layers with each other. 

%Cloud environments are composed of seven layers:facility, network, hardware, OS, middleware, application, and users~\cite{spring2011monitoring}). Understanding the interactions and correlations of different components of the cloud from different layers is complicated; we have not found a single monitoring tool that can collect operational data from all the layers of the cloud. The goal for this phase of the project is to collect virtual resource data from the cloud which can come from the OS, middleware, application and user layers of the cloud and to retain the data so that we can combine it with the physical resource data of the cloud. For example, using a performance tool, we can obtain information about a VM. This information can then be correlated with information obtained from the hypervisor regarding which physical node the VM is running.  This information can then be connected to the information collected from physical layers which shows the status of the physical nodes. 

%The research challenges for this phase are: 
%1) How to collect a large set of operational data from the cloud virtual system which can bring little overhead to the system being monitored. 
%2) How to retain or migrate the data of cloud virtual resources from a temporal database to a long-term database in a scalable manner.
%3) How to optimize the retention for data analysis.

%Although no tool can monitor all layers of the cloud, we have found a set of open source tools that are specific to each layer. We believe that we can deploy a combination of existing tools for collecting all the data we need, so no new monitoring tools will need to be built. The data collection part of this project will concern evaluating different monitoring tools that can collect data from virtual systems of the cloud. 

%A variety of different databases for retaining the data collected from the monitoring tool need to be evaluated based on criteria such as scalability and performance. We will also need to consider the migration of data that has already been collected in a default database to a single database or several databases distributed across the whole cloud.



%\subsection{Data Collection and Retention}

Cloud environments are enourmously complicated systems that are composed of seven very different layers (facility, network, hardware, OS, middleware, application, and the user~\cite{spring2011monitoring}) with complex interactions between the components and players at these different layers.  Our fundamental goal for this phase of the project is to collect information from all these different layers and retain it so that tools can be developed that can correlate information from all the different sources.  For example, a performance tool may need to connect a VM launched by a user to information from cloud middleware about what node the VM was launched on, to information from a hypervisor about what the demands of the VM were. 

% Collection of information about the cloud is done by monitoring tools that 
% There are no tools that monitor all these different layers
% with tools grouped into high-level related to cloud virtual system which is composed of middleware, application, and user layer ~\cite{Aceto2013}. Low-level monitoring is related to cloud’s physical system which is composed of the other four layers.

The research challenges for this phase are 1) how to collect operational data of the cloud from different sources in a scalable way without impacting the performance of the components being monitored, and 2)  how to retain and update the data in a scalable manner, to support multiple types of queries, and to optimize the retention for data analysis.

For data collection, while there exists no tools that collect informaton from all the different layers of the cloud, there are a wide variety of tools that are specific to different layers.  Our hypothesis for this phase is that we can use a combination of existing monitoring tools for data collection, so that we do not have to build new monitoring tools.  We will, however, need to evaluate the different alternatives from both a capability and performance perspective.  
For data retention, there are a variety of different data bases and architectures to evaluate from a scalability and performance perspective.  We will also need to consider if all the data should be incorporated into a single data base, or of information should be partitioned across a data base we define and monitoring tool specific databases. 

%In order to expose the cloud performance characteristics and utilization data to users as well as researchers, we will build a %monitoring platform that can collect and retain the performance and utilization of the cloud. The challenges for this phase %are 1) how to collect operational data of the cloud from different sources in a scalable way without impacting the performance %of the components being monitored, and 2) how to retain the data in a scalable manner and to optimize the retention for data %analysis.
%
%Cloud environment is a complicated system, and it can be modeled in seven layers: facility, network, hardware, OS, middleware, %application, and the user~\cite{spring2011monitoring}. Monitoring of these seven layers can be grouped to high-level and low-%level monitoring. 
%High-level monitoring is related to virtual cloud resources, which are composed of middleware, application, and user layer~\%cite{Aceto2013}. 
%Low-level monitoring is related to cloud’s physical infrastructure, which is composed of the other four layers. 
%
%There exists multiple successful monitoring tool options for each layer of the cloud. For data collection and retention phase %we plan to use a subset of these tools. However, as there are many options, we need to evaluate and investigate a huge set of %combinations to obtain the performance and utilization of the cloud in an effective and efficient manner. 
%
%We plan to investigate cloud data collection tools such as Sensu, Nagios, Zabbix, Ceilometer, Stacktach, and Logstash. Our %main concerns in data collection is a) having a wide range of data collection capabilities, b) being scalable, c) having a low %impact on the monitored system.  The data that we initially plan to monitor are utilization of the physical resources (e.g. %CPU, memory, disk, network, I/O bandwidth, etc...) and cloud services (e.g. authentication, cloud networking, VM image hosting %services, etc...), power consumption of components of the cloud (e.g. CPU/memory voltage, temperatures of various node %components, etc...), utilization and metering of virtual resources (e.g. how many VM hours per user, how much I/O, VM to %physical node mappings, etc…), logs from different cloud layers (e.g. OS level logs, hypervisor level logs, cloud management %service logs, etc...).   
%
%Similarly, there are multiple database options for retaining the data collected from the cloud. The database solutions we will %adopt has to be scalable, easy to update, amenable to supporting multiple types of queries. We plan to evaluate options such %as InfluxDB, MongoDB, MySQL, etc… Some data collection tools come with a database, while other monitoring tools does not %retain the data they collected. Whether to store all the data collected in one large database or store it in separate %databases as provided by some of the monitoring tools is an issue we will investigate.
%
%\subsection{Cloud Monitoring Data Visualization} 

%\subsection{Cloud Monitoring Data Transformation and Visualization}
%During this phase, we will extend the platform we built in the previous phase by filtering and transforming the retained data. Then we will provide an API  for efficient queries and an end-user tool that can visualize the status of the cloud as well as user specific utilization of the cloud.  

%The data collected from the cloud is hard to use and there are some challenges for us to filter and transform the data. First, for monitoring the same cloud, we use different monitoring tools for different levels or even layers, which means that we need a mechanism to relate data from different levels or layers. Second, the data we collected from the system is raw, which means that users may not know what does the data represents and the data need to be aggregated based on users’ demands and use cases. For example, the utilization information come from monitoring tools is compute based, we can aggregate the compute utilizations to get utilizations of a cluster. 

%Having a platform that collects and retains data that represents the status of the cloud, however, is not enough. First, as database schema may change or we may have several databases for storing all the information, direct access to database may not be an efficient way. Second, reading documentations and statistics for large set of data is not yet clear to cloud users, especially for end users. Third, exposing the data of the whole platform means risking the security of the cloud and privacy of users. So we plan to provide an API and an end-user visualization tool for showing cloud status and users’ virtual resources. 
% ************************************************************************
% ****************** Prior Work ****************************************


\section{Prior Work}
\label{sec:PriorWork}

\subsection{Data Collection and Retention}

For commercial platforms, some of the major companies such as Amazon and Microsoft, already provide monitoring services. CloudWatch, a monitoring service provided by Amazon, only have services of monitoring virtual resources such as EC2~\cite{Aceto2013}. Amazon does not provide low-level information (physical machines€™ information). AzureWatch, provided by Microsoft, also does not provide the monitoring information of low-level cloud. 

  For open source platforms, there are some monitoring tools focusing on different aspects of cloud computing infrastructures. As I focus on monitoring physical system of the cloud, I explored some monitoring tools that monitor physical systems. Nagios [6] is a widely-used enterprise-level open source monitoring tool that can monitor cloud's physical machines. Sandoval et cl.~\cite{sandoval2012evaluation} analyze several already available monitoring tools, such as Nagios, HypericHQ, Lattice, Zenoss, and indicate that Nagios is the best choice. 

  Sensu~\cite{sensu} is designed as a publish/subscribe based monitoring tool that can deal with some of the problems that Nagios cannot solve. In Nagios, there is a configuration file that needs to be modified and restart Nagios whenever we want to have any modification. However, sensu-clients, which run on remote hosts, can subscribe to some checks and can avoid restarting the whole monitoring tool. It also uses RabbitMQ, which is a message-oriented middleware that implements Advanced Message Queueing Protocol (AMQP). Sensu is said to be a monitoring tool that is scalable, extensible and elastic~\cite{Aceto2013}. 
\subsection{VM Characterization and Performance Prediction}
There have been numerous previous studies of performance prediction for optimization purpose~\cite{Yang2005} ~\cite{Matsunaga2010}. A lot of these researches are built based on data analysis and modeling techniques, depending on in-depth understanding of the application. Most of these studies come up with models of high accuracy, but these models are usually application specific ~\cite{Shan2008}. For a different application, researchers have to develop a new model and evaluate it in a specific environment. But if we can categorize any application into a certain class by clustering analysis upon same set of characteristics, user can optimize performance for a specific application and come up with some results (e.g. fix numbers of some characteristics of VM), which can be reused for scheduling and allocating VMs for some other different applications within the same class. However, there is not such a tools can categorize applications in cloud and it requires the cooperation between cloud providers, research communities and cloud users to accomplish this goal.

Recent year, many papers have been studying individual performance modeling techniques for parallel applications and systems optimization ~\cite{Brian98}~\cite{Yang2005}. All classes of individual techniques have important strengths and weaknesses. Abstract, analytical models provide significant insights into application performance and are usually extremely fast, but lack the ability to capture detailed performance utilization, and most such models must be constructed manually, which limits their accessibility to users. In contrary, program-driven simulation techniques can capture detailed performance utilization at all layers but, can be extremely expensive for large-scale parallel programs and large scale infrastructure, not only in terms of simulation time but especially in their computational resources ~\cite{Huaxia1999} ~\cite{Michael2015}. These methodologies are not perfectly applicable to modern cloud, unless the technical challenges are solved.

Another issue is data collected form cross cloud platform needs to be normalized before actual analysis. Specifically, users of high-performance computing (HPC) platforms tend to have access to more and more geographically distributed computational resources. Unfortunately, both the resources and the applications in today’s distributed computing environment are highly heterogeneous and of great complexity ~\cite{Yang2005}. This makes it difficult to measure the resource usage of a specific application on a wide range of execution platforms. Instead of translating the data set after gathering from different infrastructures, collecting data from a cross environment and storing the data in a meaningful and amenable manner will be more feasible and efficient, because it can avoid extra efforts and research resources in the normalization. MOC is providing such a heterogeneous environments, having different vendors ranging from lenovo, Cisco, Brocade, Intel, where will enable us solve this problem in a real environment.
\section{Proposed Approach}
\label{sec:ProposedApproach}

\subsection{Data Collection and Retention}

%We will first evaluate several monitoring tools in terms of scalability, elasticity, adaptability~\cite{Aceto2013}. We plan to investigate cloud data collection tools such as Sensu, Nagios, Zabbix, Ceilometer, Stacktach, and Logstash. Sensu, Nagios and Zabbix are monitoring tools for low-level monitoring of the cloud, which can collect utilization information of cloud physical system. For high-level monitoring, as MOC uses OpenStack as its platform, we plan to explore ceilometer~\cite{ceilometer}, which is an OpenStack project that can collect measurements of the utilization of virtual resources in the clouds, and StackTach, which is another tool that collects utilization of virtual resources in the cloud by  consuming notifications from OpenStack message queue. 

%Our main concerns in data collection are a) having a wide range of data collection capabilities, b) being scalable, c) having a low impact on the monitored system. The data that we initially plan to monitor are utilization of the physical resources (e.g. CPU, memory, disk, network, I/O bandwidth, etc...) and cloud services (e.g. authentication, cloud networking, VM image hosting services, etc...), power consumption of components of the cloud (e.g. CPU/memory voltage, temperatures of various node components, etc...), utilization and metering of virtual resources (e.g. how many VM hours per user, how much I/O, VM to physical node mappings, etc…), logs from different cloud layers (e.g. OS level logs, hypervisor level logs, cloud management service logs, etc...). 

%We plan to investigate multiple database options such as InfluxDB, MongoDB, MySQL, etc… MongoDB is a distributed NoSQL database which is the default database in ceilometer; InfluxDB is a time series database which incorporate with some visualization tools. 

%Our main concerns in data retention are a) being scalable in retaining and updating the data, b) being able to support multiple queries, c) how to optimize the retention for data analysis
%We need to investigate how to support multiple queries with the collected data in multiple databases, and compare the performance between using one database and multiple databases. We will investigate centralized and distributed databases so as to study the scaling behavior of retaining and updating the data. 

Our approach for realizing the data collection and retention phase of the project involves the following stages: i) evaluation of existing full-fledged solutions for monitoring different cloud layers and doing more literature analysis, ii) evaluation and comparison of different data collection and data storage solutions and determination of tools to be used, iii) implementation of data collection and retention solution. 

With the monitoring tools we mentioned in Prior Work, we will do evaluations based on the following criteria: 1) what adaptors exist for compute, storage and network; given the wide diversity of gear in the MOC it is important to adopt tools that have existing adaptors for the specific switches and compute infrastructure, 2) how much overhead does the monitoring tool impose on the monitored system, 3) how scalable is the tool, e.g. can we pair the monitoring tool with scalable distributed databases, and 4) is the monitoring tool compatible with other monitoring goals of the MOC, e.g., ability to send failure alerts to administrators, enabling metering and billing solutions, etc.

We plan to investigate multiple database options such as InfluxDB, MongoDB, MySQL, etc... MongoDB is a distributed NoSQL database, which is the default database used by ceilometer; InfluxDB is a time series database, which incorporates well with visualization tools such as Grafana. Our main concerns in data retention are: a) having support for scalable data storage and update, b) having support for multiple types of queries.

We also need to investigate how to support queries that run on multiple databases, and compare the performance between using one database and multiple databases. We will investigate centralized and distributed databases so as to study the scaling behavior of retaining and updating the data. 


\subsection{Cloud Monitoring Data Visualization}
  
During this phase, we will extend the platform we built in last phase by filtering and transforming the retained data and by providing an API  for efficient queries and an end-user tool that can visualize the status of the cloud as well as user specific utilization of the cloud. Along with the project, we will put effort in solving the problems we proposed in problem statement section. 

The first challenge is how to ensure scalability for serving many queries. It is expensive to do queries in large data set. Cui et al. ~\cite{cuiusing} state that without optimizations, complex queries, such as correlation detection, on large time series data take tens of minutes to be done. So we need to do proper filtering and transforming with the data collected from the cloud in order to make more efficient queries. We will first research into this field where optimizations have been made in databases in order to support scalability. As for a start, we plan to aggregate single compute node utilizations to the whole cluster’s compute node utilization and to aggregate  instantaneous utilization data to average utilization data within a time range. 

Second, we need to correlate data from different layers monitored by different monitoring tools. The data from some layers are more useful to users and researchers when they are correlated with the data from other layers. For example, information of physical machines’ conditions will be more utilizable to users if they are correlated with users’ virtual machines information. So we will develop a mapping of users’ virtual resources and physical resources that they utilized. 

Third, we need to decide to what extent we can expose the information and also protect the security of the cloud. It would be a threat to the cloud if we grant everyone direct access to databases, and also it would not be an efficient way, for the reason that we may have several databases and we will have a large data set. We will develop a user API to allow users to query the data in databases with the queries pre-defined by us. The queries will be carefully designed through the studying of users’ demands (e.g: price, performance, energy efficiency and time,etc…).

Fourth, we need to protect user privacy while still providing valuable information. As stated above, we will have a mapping of users’ virtual resources with physical resources they utilized. End users can only see the information of their own projects, which contains information of physical resources that are related to their projects only, and they may need to sign agreements that they won’t share the details with other users. We will need to write programs that can talk to authentication services, such as Keystone in OpenStack. 

Last but not the least, to exercise and refine the platform, we will develop an end-user tool that visualizes both the status of the cloud, and the users virtual cloud resources. As there are many metrics of the cloud to show, and the data can be instantaneous as well as historical, we need to develop a solution that show the data in a clear form and also provide valuable information to users. We will also explore open source visualization tools, such as Grafana~\cite{grafana}, which is a  visualizing tool and can analyze data within a time range.  




\section{Plan and Schedule}
\label{sec:Plan}

\begin{itemize}
\item{[Sep 2015 - Oct 2015]:} Consider many use cases for users of MOC and come up with a list of metrics that are useful for users about MOC. Use existing monitoring tool in MOC, Sensu, to get physical system information of MOC. Explore other monitoring tools that can get physical system information can evaluate them with Sensu. Collect the information as need from the monitoring tool we chose.
\item{[Oct 2015 - Nov 2015]:} Evaluate metering tools, which are used for calculating how much resource a tenant has used. The options now are ceilometer and StackTach. We need to consider the overhead and quantity of the messages. The we will choose a better choice as our metering tool for mappings. Installation and setup of the metering tool that we chose after evaluation. Write a program for mappings for each project using the metering tool we chose after evaluation.
\item{[Nov 2015 - Dec 2015]:} Explore the data that we can get from network switches, and write programs that can show how the physical hosts are connected via networks. 
\item{[Dec 2015 - Jan 2016]:} Evaluate databases and decide if we should use multiple databases or just one database. If we choose one database, we need to come up with a schema for storing all the data. If we choose multiple databases, we need to come up with a way to have a good connection of the data. 
\item{[Jan 2016 - Feb 2016]:} Using the data we collect and stored in the database, we need to make a visualization tool that can visualize both physical system information and a mapping between physical resources and virtual resources for each user. 
\item{[Feb 2016 - Mar 2016]:} Continue making the visualization tool and also come up with use cases for users and design the API for them. 
\item{[Mar 2016 - Apr 2016]:} Write programs that can provide API for users. 
\item{[Apr 2016 - Jun 2016]:} Refine all parts of the project. Write the report.
\end{itemize}


\bibliographystyle{ieeetr}
\bibliography{main}

\end{document}
