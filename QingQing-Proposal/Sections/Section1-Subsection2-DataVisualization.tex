\subsection{Data Transformation and Visualization} 

One of the main arguments for utilizing public clouds is it is an environment where developers with innovative ideas can try out their new services without investing in huge amounts of hardware~\cite{Armbrust2009}. However, public cloud providers don't expose the performance characteristics and utilization of the cloud software and hardware components to cloud users and researchers. As a result, developers are often forced to do reverse engineering to discover the characteristics of the cloud and optimize their applications. A good example to this is presented in~\cite{conley2015achieving}, where Conley et al. try to identify the best performing instance type for sorting 100TB of data on Amazon EC2. After evaluating 48 different instance types available, and spending \$50.000 along the way, they identified the best performing instance type. That instance type can sort 100TB of data at a cost of \$300. 

Unfortunately, even the authors of~\cite{conley2015achieving} are not confident that their experiments are repeatable as the utilization of the cloud and the resources they used when they performed these experiments are unknown to them. Our fundamental goal in this project is to expose the kind of information that Conley at al. spent enormous effort and money trying to discern from the outside. 

There are many valid reasons why cloud providers do not want to provide visibility into the performance and utilization characteristics of their cloud infrastructure. First, they want to keep their solutions private due to commercial concerns. Second, sharing details of the underlying cloud infrastructure may lead to security problems. Third, providing information about a user's resources may expose information about other users in the cloud and cause privacy issues. 

Since we plan  to implement our monitoring solution on an open cloud setting (MOC), we are not bound by commercial concerns. Since no public cloud vendor has ever exposed the cloud performance characteristic and utilization data to users and researchers, we are facing a new area with challenges such as security issues and privacy issues that have not been studied by researchers.

The research challenges for this phase are 1) how to make the platform scalable so as to serve many queries, 2) what kind of queries should we have to ensure privacy while providing valuable information to users, 3) to what extent should we expose the information as to ensure security of the cloud system, and 4) how to relate the data of different cloud layers with each other. 

