\section{Problem Statement}
\label{sec:ProblemStatement}

%\subsection{Introduction}

One of the main arguments for utilizing public clouds is it is an environment where developers with innovative ideas can try out their new services without investing in huge amounts of hardware~\cite{Armbrust2009}. However, public cloud providers refrain from exposing the performance characteristics and utilization of the cloud software and hardware components from cloud users and researchers. As a result, developers are often forced to do reverse engineering to discover the characteristics of the cloud and optimize their applications. A good example to this is presented in~\cite{conley2015achieving}, where Conley et al. try to identify the best performing instance type for sorting 100TB of data on Amazon EC2. After evaluating 48 different instance types available, and spending \$50.000 along the way, they identified the best performing instance type. That instance type can sort 100TB of data at a cost of \$300. 

Unfortunately, even the authors of~\cite{conley2015achieving} are not exactly sure that their experiments are repeatable as the utilization of the cloud and the resources they used when they performed these experiments are unknown to them. Our goal in this project is to prevent this kind of performance uncertainty by providing the underlying infrastructure utilization information to users so that they can do intelligent decisions.          

However, there are many valid reasons why cloud providers do not want to provide visibility into the performance and utilization characteristics of their cloud infrastructure. First, they want to keep their solutions private due to commercial concerns. Second, sharing details of the underlying cloud infrastructure may lead to security problems. Third, providing information about a user?s resources may expose information about other users in the cloud and cause privacy issues. 

On the other hand, since we envision to implement our monitoring solution on an open cloud setting (MOC), we are not bound by commercial concerns. Also, we believe that security through obscurity is not the right solution for achieving high level cloud security and providing increased transparency to the cloud infrastructure will foster building of more secure cloud solutions. Finally, one of our goals in this project is to come up with ways of exposing cloud utilization information in a manner that preserves user privacy.

This project will be composed of two phases. The first phase, Data Collection and Retention, will be a platform building phase that collects and stores raw data from the MOC. It will be a shared effort with another master student in his project--'Open Cloud for Cloud Research'.  
The second phase, Data Transformation and Visualization, will first extend the platform by filtering and transforming the raw data collected from the cloud such that a number of query sets can be efficiently performed over the data. We will then exercise and refine the platform by developing an end-user tool that visualizes both the status of the cloud, and the user’s virtual cloud resources. The goal of this tool will be to provide users the kind of information they have to reverse engineer on existing clouds to optimize their applications.

\subsection{Cloud Monitoring Data Collection and Retention}
\subsection{Cloud Monitoring Data Visualization} 

%\subsection{Cloud Monitoring Data Transformation and Visualization}
%During this phase, we will extend the platform we built in the previous phase by filtering and transforming the retained data. Then we will provide an API  for efficient queries and an end-user tool that can visualize the status of the cloud as well as user specific utilization of the cloud.  

%The data collected from the cloud is hard to use and there are some challenges for us to filter and transform the data. First, for monitoring the same cloud, we use different monitoring tools for different levels or even layers, which means that we need a mechanism to relate data from different levels or layers. Second, the data we collected from the system is raw, which means that users may not know what does the data represents and the data need to be aggregated based on users’ demands and use cases. For example, the utilization information come from monitoring tools is compute based, we can aggregate the compute utilizations to get utilizations of a cluster. 

%Having a platform that collects and retains data that represents the status of the cloud, however, is not enough. First, as database schema may change or we may have several databases for storing all the information, direct access to database may not be an efficient way. Second, reading documentations and statistics for large set of data is not yet clear to cloud users, especially for end users. Third, exposing the data of the whole platform means risking the security of the cloud and privacy of users. So we plan to provide an API and an end-user visualization tool for showing cloud status and users’ virtual resources. 