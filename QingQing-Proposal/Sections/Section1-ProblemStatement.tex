\section{Problem Statement}
\label{sec:ProblemStatement}

A major problem in Cloud Computing is the lack of visibility for cloud users and researchers into the performance charactistics and utilization of the cloud software and hardware components. The opaque nature of today’s public clouds result in developers that want to tune their applications spending enormous time and money on reverse engineering what is going on in the underlying cloud. In addition, researchers are forced to innovate with limited information about the actual demands on today’s clouds. In this project we will: 1) develop a platform that collects and retains rich information about a cloud, 2) enhance the platform to transform and expose the information in a way that it can be used for a range of use cases, and 3) develop a tool that exercises the platform to provide the users of the cloud information necessary to tune their applications. This work will be done in the context of the Massachusetts Open Cloud (MOC), a new public cloud project that has the central goal of enablign open cloud research. Since the MOC is developed at BU, if this research is successful, it will be integrated into the large-scale production offering that we expect will be used by hundreds if not thousands of users.  

\subsection{Cloud Monitoring Data Collection and Retention}

\subsection{Cloud Monitoring Data Visualization}
  We propose that exposing operational data of the cloud in a readable, organized and meaningful way can help users in making decisions. The operational data collected from the cloud and stored databases, however, is rather hard to use, especially for end users. First, the operational data we collected come from various sources, such as logs, monitoring tools, and metering tools. Second, the data we collected from the system is raw and need to be aggregated in order to be represented in a meaningful way. Third, as the database schema may change or we may have several databases for storing all the information, direct access to database may not be a efficient way. Fourth, people sometimes get tired of reading all the documentations and reading all the statistics. On the contrary, they would rather have a visual access to all data as needed. Last but not the least, exposing the whole database to users may cause some security problem and information leakage. Also, we intend to expose the data in a way that each user can only view the usages and conditions of its own resources.  

  In order to expose cloud system's operational data to users in a readable, organized and meaningful manner, this Data Visualization part emphasize in dealing with the problems stated above. We will complete the platform by  maitaining a larger set of data adn doing efficient data filtering and data aggregation. In order to let users do queries more efficiently, we will provide an API in a programmable mannerbased on the metrics that users will need for exploring the resource and researching purposes. In order to expose the data in a readable manner, we will develop a visualization tool for users to have a clear understanding of the cloud and their resources. This part will focus in the monitored data from low-level cloud and also a mapping between low-level cloud and high-level cloud, which represents the virtual system of the cloud. In order to provide a certain level of privacy and security, each user will have access to the data of physical machines that his or her virtual resources reside in. 