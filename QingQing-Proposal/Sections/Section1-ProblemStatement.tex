\section{Problem Statement}
\label{sec:ProblemStatement}

This project will be composed to two phases. The first phase, Data Collection and Retention, will be a platform building phase that collects and stores raw data from the MOC. It will be a shared effort with another master student in his project--'Open Cloud for Cloud Research'.  
The second part, Data Transformation and Visualization, will first extend the platform by filtering and transforming the raw data collected from the cloud to the data that direct queries can be done with and provide users with an API for efficient queries of the data. We will then exercise and refine the platform by developing an end-user tool that visualizes both the status of the cloud, and the user’s virtual cloud resources. The goal of this tool is to provide users the kind of information they have to reverse engineer on existing clouds to optimize their applications.

\subsection{Cloud Monitoring Data Collection and Retention}

\subsection{Cloud Monitoring Data Visualization}
  We propose that exposing operational data of the cloud in a readable, organized and meaningful way can help users in making decisions. The operational data collected from the cloud and stored databases, however, is rather hard to use, especially for end users. First, the operational data we collected come from various sources, such as logs, monitoring tools, and metering tools. Second, the data we collected from the system is raw and need to be aggregated in order to be represented in a meaningful way. Third, as the database schema may change or we may have several databases for storing all the information, direct access to database may not be a efficient way. Fourth, people sometimes get tired of reading all the documentations and reading all the statistics. On the contrary, they would rather have a visual access to all data as needed. Last but not the least, exposing the whole database to users may cause some security problem and information leakage. Also, we intend to expose the data in a way that each user can only view the usages and conditions of its own resources.  

  In order to expose cloud system's operational data to users in a readable, organized and meaningful manner, this Data Visualization part emphasize in dealing with the problems stated above. We will complete the platform by  maitaining a larger set of data adn doing efficient data filtering and data aggregation. In order to let users do queries more efficiently, we will provide an API in a programmable mannerbased on the metrics that users will need for exploring the resource and researching purposes. In order to expose the data in a readable manner, we will develop a visualization tool for users to have a clear understanding of the cloud and their resources. This part will focus in the monitored data from low-level cloud and also a mapping between low-level cloud and high-level cloud, which represents the virtual system of the cloud. In order to provide a certain level of privacy and security, each user will have access to the data of physical machines that his or her virtual resources reside in. 