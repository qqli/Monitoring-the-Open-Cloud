\section{Problem Statement}
\label{sec:ProblemStatement}

\subsection{Cloud Monitoring Data Collection and Retention}

  \subsection{Cloud Monitoring Data Visualization}
  The operational data collected from the cloud and stored databases, however, is rather hard to use, especially for end users. First, the operational data we collected come from various sources, such as logs, monitoring tools, and metering tools. Second, the data we collected from the system is raw and need to be aggregated in order to be represented in a meaningful way. Third, as the database schema may change or we may have several databases for storing all the information, direct access to database may not be a efficient way. Fourth, people sometimes get tired of reading all the documentations and reading all the statistics. On the contrary, they would rather have a visual access to all data as needed. Last but not the least, exposing the whole database to users may cause some security problem and information leakage. Also, we intend to expose the data in a way that each user can only view the usages and conditions of its own resources.  
  In order to expose cloud system operational data to users in a readable, organized and meaningful manner, this Data Visualization part emphasize in dealing with the problems stated above. We will complete the platform by  maitaining a larger set of data adn doing efficient data filtering and data aggregation. In order to let users do queries more efficiently, we will provide an API in a programmable mannerbased on the metrics that users will need for exploring the resource and researching purposes. We will also provide a visualization as well as APIs for users. In order to provide the data in a readable manner, we will develop a visualization tool for users to have a clear understanding of the cloud and their resources. 