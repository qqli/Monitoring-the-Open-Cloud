\section{Proposed Approach}
\label{sec:ProposedApproach}
To achieve our project goals, which is exposing clouds’ operational data to users and provide them a visualization of the data, we can follow several steps. The first is to decide the set of data that are useful to be collected. The second is how to collect that information. The third is how to retain that information. 
  First, we need to come up with a list of metrics that are useful for users about MOC. We know that user will be concerned with time, performance, price, energy efficiency when they use MOC.  A cloud physical system can be divided into three parts: compute, storage and network. For each of the component in the cloud, we need to find parameters that can influence the factors that users are concerned with, such as performance. Because cloud is a large system, it can have thousands of physical machines and storages and complex networks. Storing data that are less useful seem to be a waste of storage and can lower the query speed. So we need to do research in papers, and find out users’ demands when they consider using a cloud. Based on users’ demand information, we can come up with a list of metrics that are useful to collect. Furthermore, we need also consider the scope of the data. For example, whether the data, such as cpu utilization, should be compute based or cluster based. Or if we should also store some historical data such as hourly cpu utilization or daily cpu utilization of a compute node. 
  In order to collect the data, we need to evaluate several monitoring tools. This project mainly focuses on physical system monitoring and a mapping between physical resources and virtual resources for users. From Prior Work section, we know that Sensu, Nagios and a lot of monitoring tools can collect the information of cloud’s physical system. We need to compare the difference between these monitoring tools. Monitoring of network switches, which are not commonly implemented in monitoring tools, is also needed to explore. 
  Apart from low-level monitoring, we still need some high-level monitoring to provide a mapping between the two for users. MOC uses OpenStack[4] as its platform to provision VMs and storage units. OpenStack has a generic metering project called ceilometer[18]. We need to install and test ceilometer and compare ceilometer with other metering tools, such as StackTach, which can provide us with the same information. 
  The third step is to retain the data in a meaningful and organized way. We need to evaluate many databases that can be used for our project and also consider how we can retain all the data. As cloud is a complex and huge system, we need to choose a database that is able to store large data sets. For database selection, we have to evaluate several databases based on its scalability and speed. We also need to explore the distributed characteristics databases, as this can increase fault-tolerance, which is desired in a cloud environment. A reasonable database schema is also needed to store all the data. Moreover, as we have potential data from logs, monitoring tools, and metering tools, all of them may have its own database. To study their database and decide if we need to put all the collected in a single database or distribute them among different database is critical to our research. 
  Fourth, with the data collected and stored in the database/databases, we will make a data visualization tool. We need to explore how to present the data in a fashionable and readable way. This data visualization tool should show basic descriptions of the cloud, instantaneous usage of the cloud, and the mapping of physical machines and virtual machines. We will also explore open source visualization tools, such as Grafana [14] for visualizing and analyzing data within a time range. 

