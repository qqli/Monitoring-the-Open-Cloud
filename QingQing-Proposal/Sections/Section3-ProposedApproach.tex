\section{Proposed Approach}
\label{sec:ProposedApproach}

To achieve our project goals, we will start out our project with a shared part, which is the platform building. In order to expose data to users for my project and to categorize VMs for Hua?s project, we need to collect and retain operational data from the cloud. 

  In order to collect the data, we need to evaluate several monitoring tools in terms of scalability, elasticity, adaptability [cloud monitoring: a survey]. This project will mainly focus on physical system monitoring and usage of virtual resources. From some literatures, we learn that Sensu, Nagios and a lot of monitoring tools can collect the information of cloud��s physical system. Apart from low-level monitoring, we still need some high-level monitoring to provide virtual resources utilization to users. MOC uses OpenStack[4] as its platform to provision VMs and storage units. OpenStack has a generic metering project called ceilometer[18]. We need to install and test ceilometer and compare ceilometer with other metering tools, such as StackTach. Additionally, this project will investigate monitoring of network switches, which are not a common features of most monitoring tools. 
  
  Data collection is another big challenge this project needs to address for platform building. As monitoring event produce huge amount of data, so the storage must be scalable to support efficient queries of massive amount of data in the future. We also need to explore characteristics of distributed databases, as this can increase fault-tolerance, which is desired in a cloud environment. And the data schema needs to be defined in an normalized form amenable to data analysis. Moreover, as we have potential data from logs, monitoring tools, and metering tools, all of them may have its own best fit for storage. Retaining data in a single database or distributing them among different database is one critical question we will study. So far we have two candidates, MongoDB and InfluxDB. MongoDB is a mature distributed NoSQL database while InfluxDB is a time series database. We will evaluate them by looking into literatures and trialing on them in terms of all issues.
  
  After building the platform that is functioned to collect and retain certain amount of data. Hua and I will focus on different parts of the project. I will separate my part of the project to three sections. First, as the project goes along, we will get larger set of data. This part will focus in maintaining the data collected and store it in a way that is well organized and optimized for querying. Storing data that are less useful seem to be a waste of storage and can lower the query speed. So we will do literature reviews, and find out demands of users when they consider using a cloud. Based on users? demand information and users concern such as prices, performance, energy efficiency and time, we can come up with a list of metrics that are useful to collect. For each of the component in the cloud, we need to find parameters that can influence the factors that users are concerned with, such as performance. Furthermore, we need store some useful aggregations of data. People sometimes are more concerned about the usage over a period of time than just instantaneous data. However, monitoring tools usually get instantaneous data only. We will define a schema that can store large set of historical data such as hourly cpu utilization or daily cpu utilization of a compute node.
  
  Second, after retaining all the useful information in one or more database, we need to expose instantaneous cloud operational data  and some historical data as stated above to users via API in a programmable manner. Literature reviews are needed to explore the queries users want to have when they are using the cloud. For example, we will write an API that can provide users a mapping between their virtual resources and cloud physical resources. This particular API will tell them conditions of  the hosts for their VMs and also present them the physical networks of their VMs. 
  
  Third, with the data collected and stored in the database/databases, we will make a data visualization tool. We need to explore how to present the data in a fashionable and readable way. This data visualization tool should show basic descriptions of the cloud, instantaneous usage of the cloud, and the mapping of physical machines and virtual machines for users. We will also explore open source visualization tools, such as Grafana [14] for visualizing and analyzing data within a time range.


