\section{Proposed Approach}
\label{sec:ProposedApproach}

\subsection{Data Collection and Retention}

%We will first evaluate several monitoring tools in terms of scalability, elasticity, adaptability~\cite{Aceto2013}. We plan to investigate cloud data collection tools such as Sensu, Nagios, Zabbix, Ceilometer, Stacktach, and Logstash. Sensu, Nagios and Zabbix are monitoring tools for low-level monitoring of the cloud, which can collect utilization information of cloud physical system. For high-level monitoring, as MOC uses OpenStack as its platform, we plan to explore ceilometer~\cite{ceilometer}, which is an OpenStack project that can collect measurements of the utilization of virtual resources in the clouds, and StackTach, which is another tool that collects utilization of virtual resources in the cloud by  consuming notifications from OpenStack message queue. 

%Our main concerns in data collection are a) having a wide range of data collection capabilities, b) being scalable, c) having a low impact on the monitored system. The data that we initially plan to monitor are utilization of the physical resources (e.g. CPU, memory, disk, network, I/O bandwidth, etc...) and cloud services (e.g. authentication, cloud networking, VM image hosting services, etc...), power consumption of components of the cloud (e.g. CPU/memory voltage, temperatures of various node components, etc...), utilization and metering of virtual resources (e.g. how many VM hours per user, how much I/O, VM to physical node mappings, etc…), logs from different cloud layers (e.g. OS level logs, hypervisor level logs, cloud management service logs, etc...). 

%We plan to investigate multiple database options such as InfluxDB, MongoDB, MySQL, etc… MongoDB is a distributed NoSQL database which is the default database in ceilometer; InfluxDB is a time series database which incorporate with some visualization tools. 

%Our main concerns in data retention are a) being scalable in retaining and updating the data, b) being able to support multiple queries, c) how to optimize the retention for data analysis
%We need to investigate how to support multiple queries with the collected data in multiple databases, and compare the performance between using one database and multiple databases. We will investigate centralized and distributed databases so as to study the scaling behavior of retaining and updating the data. 

Our approach for realizing the data collection and retention phase of the project involves the following stages: i) evaluation of existing full-fledged solutions for monitoring different cloud layers and doing more literature analysis, ii) evaluation and comparison of different data collection and data storage solutions and determination of tools to be used, iii) implementation of data collection and retention solution. 

With the monitoring tools we mentioned in Prior Work, we will do evaluations based on the following criteria: 1) what adaptors exist for compute, storage and network; given the wide diversity of gear in the MOC it is important to adopt tools that have existing adaptors for the specific switches and compute infrastructure, 2) how much overhead does the monitoring tool impose on the monitored system, 3) how scalable is the tool, e.g. can we pair the monitoring tool with scalable distributed databases, and 4) is the monitoring tool compatible with other monitoring goals of the MOC, e.g., ability to send failure alerts to administrators, enabling metering and billing solutions, etc.

We plan to investigate multiple database options such as InfluxDB, MongoDB, MySQL, etc... MongoDB is a distributed NoSQL database, which is the default database used by ceilometer; InfluxDB is a time series database, which incorporates well with visualization tools such as Grafana. Our main concerns in data retention are: a) having support for scalable data storage and update, b) having support for multiple types of queries.

We also need to investigate how to support queries that run on multiple databases, and compare the performance between using one database and multiple databases. We will investigate centralized and distributed databases so as to study the scaling behavior of retaining and updating the data. 


\subsection{Cloud Monitoring Data Visualization}
  
  After building the platform that is functioned to collect and retain certain amount of data. We will work in exposing data that is clear and useful to users. As the raw data generated from the running cloud come from different sources and maybe stored in different databases, we will first aggregate the data collected from MOC. As cloud is a complex system, even physical system has many layers. Many compute nodes forms a rack, racks of computes and storage system forms a cluster, and a cloud can have many clusters. The data we collected is mainly at compute node level. However, if we want to expose the data more meaningfully, we need to aggregate it to cluster level. In addition, people sometimes are more concerned about the usage over a period of time than just instantaneous data. However, monitoring tools usually get instantaneous data only. We will define a schema that can store large set of historical data such as hourly cpu utilization or daily cpu utilization of a compute node. Then we will aggregate the data based on users' needs, we will do literature reviews, and find out demands of users when they consider to use a cloud. Based on users' demands information, such as price, performance, energy efficiency and time, we can come up with a list of metrics that are useful to aggregate with. We can also store these aggregations to the database for later use.
  
  Second, after retaining all the useful information in one or more database, we need to expose instantaneous cloud operational data  and some historical data as stated above to users via API in a programmable manner. Literature reviews are needed to explore the queries users want to have when they are using the cloud. For example, we will write an API that can provide users a mapping between their virtual resources and cloud physical resources. This particular API will tell them conditions of  the hosts for their VMs and also present them the physical networks of their VMs. 
  
  Third, with the data collected and stored in the database/databases, we will make a data visualization tool. We need to explore how to present the data in a fashionable and readable way. This data visualization tool should show basic descriptions of the cloud, instantaneous usage of the cloud, and the mapping of physical machines and virtual machines for users. We will also explore open source visualization tools, such as Grafana [14] for visualizing and analyzing data within a time range.


