% ************************************************************************
% ****************** Prior Work ****************************************


\section{Prior Work}
\label{sec:PriorWork}

\subsection{Data Collection and Retention}

For commercial platforms, some of the major companies such as Amazon and Microsoft, already provide monitoring services. CloudWatch, a monitoring service provided by Amazon, only have services of monitoring virtual resources such as EC2~\cite{Aceto2013}. Amazon does not provide low-level information (physical machines€™ information). AzureWatch, provided by Microsoft, also does not provide the monitoring information of low-level cloud. 

  For open source platforms, there are some monitoring tools focusing on different aspects of cloud computing infrastructures. As I focus on monitoring physical system of the cloud, I explored some monitoring tools that monitor physical systems. Nagios [6] is a widely-used enterprise-level open source monitoring tool that can monitor cloud's physical machines. Sandoval et cl.~\cite{sandoval2012evaluation} analyze several already available monitoring tools, such as Nagios, HypericHQ, Lattice, Zenoss, and indicate that Nagios is the best choice. 

  Sensu~\cite{sensu} is designed as a publish/subscribe based monitoring tool that can deal with some of the problems that Nagios cannot solve. In Nagios, there is a configuration file that needs to be modified and restart Nagios whenever we want to have any modification. However, sensu-clients, which run on remote hosts, can subscribe to some checks and can avoid restarting the whole monitoring tool. It also uses RabbitMQ, which is a message-oriented middleware that implements Advanced Message Queueing Protocol (AMQP). Sensu is said to be a monitoring tool that is scalable, extensible and elastic~\cite{Aceto2013}. 

\subsection{Cloud Monitoring Data Transformation and Visualization}

  We also explore some open source solutions that are similar to our idea of monitoring the cloud and exposing the information to users. PCMONS ~\cite{chaves2011toward} is an open source pull-based monitoring solution. However, it has three levels of monitoring agents for physical level monitoring, which can cause network congestion. Moreover, it does not provide a mapping between virtual resources and physical resources. 
  GMonE~\cite{montes2013gmone} is a monitoring tool that can both monitor physical system and virtual system. However, it doesn't specify it can provide a mapping between physical systems and virtual systems. In addition, this solution has excessive use of monitoring agents and databases. It imposes the installation of agents on each VM and a database for each user. 
  DARGOS~\cite{povedano2013dargos}  is an agent-based, publish/subscribe monitoring solution for multi-tenant clouds. Administrators can see the whole cloud and each tenant has its own view of the cloud. It supports several distributed databases set up across the cloud, which enables scalability and fault-tolerance. MonPaaS~\cite{alcaraz2015monpaas} is a monitoring tool that provides monitoring functions similar to DARGOS, information of physical as well as virtual resources and a mapping between virtual and physical resources. However, as we use OpenStack as our platform, DARGOS and MonPaaS will then not be a perfect solution for our  platform.