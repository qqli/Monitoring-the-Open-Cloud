% ************************************************************************
% ****************** Prior Work ****************************************


\section{Prior Work}
\label{sec:PriorWork}

%\subsection{Data Collection and Retention}

For commercial platforms, some of the major companies such as Amazon and Microsoft, already provide monitoring services. CloudWatch, a monitoring service provided by Amazon, only have services of monitoring virtual resources such as EC2~\cite{Aceto2013}. Amazon does not provide low-level information (physical machines€™ information). AzureWatch, provided by Microsoft, also does not provide the monitoring information of low-level cloud. 

  For open source platforms, there are some monitoring tools focusing on different aspects of cloud computing infrastructures. As I focus on monitoring physical system of the cloud, I explored some monitoring tools that monitor physical systems. Nagios [6] is a widely-used enterprise-level open source monitoring tool that can monitor cloud's physical machines. Sandoval et cl.~\cite{sandoval2012evaluation} analyze several already available monitoring tools, such as Nagios, HypericHQ, Lattice, Zenoss, and indicate that Nagios is the best choice. 

  Sensu~\cite{sensu} is designed as a publish/subscribe based monitoring tool that can deal with some of the problems that Nagios cannot solve. In Nagios, there is a configuration file that needs to be modified and restart Nagios whenever we want to have any modification. However, sensu-clients, which run on remote hosts, can subscribe to some checks and can avoid restarting the whole monitoring tool. It also uses RabbitMQ, which is a message-oriented middleware that implements Advanced Message Queueing Protocol (AMQP). Sensu is said to be a monitoring tool that is scalable, extensible and elastic~\cite{Aceto2013}. 
%\subsection{Data Collection and Retention}

Several research tools have been developed to collect and to retain information across all layers of  a cloud~\cite{chaves2011toward,montes2013gmone,povedano2013dargos,alcaraz2015monpaas}.  The focus of these projects have been on specific end tools for users or administrators of the cloud. Meanwhile the primary goal of this part of our project is to develop a platform that can be used by a wide variety of tools. Specific limitations of these tools from a data collection and retention perspective include:  PCMONS~\cite{chaves2011toward} does not collect information needed to map virtual resources to physical resources. GMonE~\cite{montes2013gmone} is not appropriate for a general purpose IaaS cloud like the MOC because  it installs agents on each VM and a database for each user.  We are still evaluating what components and architectural elements from DARGOS~\cite{povedano2013dargos} can be used, although there is some concern about the non-scalable database. 

Since MOC infrastructure relies heavily on OpenStack~\cite{sefraoui2012openstack}, we also plan to investigate monitoring solutions available in OpenStack. We will evaluate two tools, Ceilometer~\cite{ceilometer} and  StackTach~\cite{stacktach}, based on scalability and performance impact . Ceilometer is OpenStack’s metering solution and there are major concerns about its performance. Ceilometer stores its data in MongoDB~\cite{mongodb} by default. StackTach~\cite{stacktach} is not directly an OpenStack service, but still an alternative to Ceilometer for metering and billing.

There exists commercial solutions running on top of public clouds that visualize cloud usage. For example, Amazon CloudWatch is a monitoring service provided by Amazon. It can monitor virtual resources such as Amazon EC2 instances, Amazon DynamoDB tables, etc., but it does not provide physical resources monitoring services. 

Academic studies that provide more complete monitoring solutions~\cite{chaves2011toward, montes2013gmone, povedano2013dargos, alcaraz2015monpaas} generally focus on either providing enhanced solution to cloud providers or provide solutions that do not carry the concerns we have such as being privacy preserving.