\documentclass[11pt,draft,letterpaper,compsoc]{IEEEtran}
\usepackage{enumerate}
\usepackage{color}
\usepackage[dvipdf,dvips]{graphicx}
\usepackage{epstopdf}
\usepackage{subfigure}
\usepackage[cmex10]{amsmath}
\usepackage{algorithm}
\usepackage{algorithmic}
\usepackage{url}

\hyphenation{op-tical net-works semi-conduc-tor}

\begin{document}
\title{BU ECE Dept. MSc Proposal\\ CloudWatch: An Open Cloud Monitoring Environment}

\author{Qingqing~Li (U20626734) \\ (Advisor(s): Prof.~Orran~Krieger, Dr. Ata Turk)\\ [8mm]Student Signature:\\ [8mm] Advisor Signature: \\ [8mm]}

\IEEEcompsoctitleabstractindextext{%
\begin{abstract}

Cloud computing has many intrinsic advantages as well as some challenges, such as performance unpredictability. Public cloud, which is usually owned by a single provider, is rather opaque to users. It does not provide information at hardware level to users, which restricts users' decision making when experimenting in clouds. This kind of data can only be obtained through monitoring the cloud and carefully retaining the data. We will use Massachusetts Open Cloud (MOC) as our cloud environment. MOC is the first cloud that aims at letting many vendors participate in a cloud, in which monitoring can be more meaningful, for the reason that users will have more choices, such as brands, in MOC environment. We propose that exposing some extent of data to users can help them to make wise choices. This project will be separated to two parts. The first part, Data Collection and Retention, will be a platform building phase that collects and stores raw data generated from MOC. It will be a two-month work with another master student in his project--'Open Cloud for Cloud Research'. The second part, Data Visualization, will be the completion of the platform by filtering and exposing the data collected in a readable and meaningful way. We will provide users with an API for efficient queries of the data. A visualization that shows the MOC's physical information and also a specific mapping between physical resources and virtual resources for each user.

\end{abstract}

\begin{IEEEkeywords}
    Monitoring, Cloud, MOC, Visualization.
\end{IEEEkeywords}}

% make the title area
\maketitle

\IEEEdisplaynotcompsoctitleabstractindextext

\IEEEpeerreviewmaketitle

\section{Problem Statement}
\label{sec:ProblemStatement}

%\subsection{Introduction}

One of the main arguments for utilizing public clouds is it is an environment where developers with innovative ideas can try out their new services without investing in huge amounts of hardware~\cite{Armbrust2009}. However, public cloud providers refrain from exposing the performance characteristics and utilization of the cloud software and hardware components from cloud users and researchers. As a result, developers are often forced to do reverse engineering to discover the characteristics of the cloud and optimize their applications. A good example to this is presented in~\cite{conley2015achieving}, where Conley et al. try to identify the best performing instance type for sorting 100TB of data on Amazon EC2. After evaluating 48 different instance types available, and spending \$50.000 along the way, they identified the best performing instance type. That instance type can sort 100TB of data at a cost of \$300. 

Unfortunately, even the authors of~\cite{conley2015achieving} are not exactly sure that their experiments are repeatable as the utilization of the cloud and the resources they used when they performed these experiments are unknown to them. Our goal in this project is to prevent this kind of performance uncertainty by providing the underlying infrastructure utilization information to users so that they can do intelligent decisions.          

However, there are many valid reasons why cloud providers do not want to provide visibility into the performance and utilization characteristics of their cloud infrastructure. First, they want to keep their solutions private due to commercial concerns. Second, sharing details of the underlying cloud infrastructure may lead to security problems. Third, providing information about a user?s resources may expose information about other users in the cloud and cause privacy issues. 

On the other hand, since we envision to implement our monitoring solution on an open cloud setting (MOC), we are not bound by commercial concerns. Also, we believe that security through obscurity is not the right solution for achieving high level cloud security and providing increased transparency to the cloud infrastructure will foster building of more secure cloud solutions. Finally, one of our goals in this project is to come up with ways of exposing cloud utilization information in a manner that preserves user privacy.

This project will be composed of two phases. The first phase, Data Collection and Retention, will be a platform building phase that collects and stores raw data from the MOC. It will be a shared effort with another master student in his project--'Open Cloud for Cloud Research'.  
The second phase, Data Transformation and Visualization, will first extend the platform by filtering and transforming the raw data collected from the cloud such that a number of query sets can be efficiently performed over the data. We will then exercise and refine the platform by developing an end-user tool that visualizes both the status of the cloud, and the user’s virtual cloud resources. The goal of this tool will be to provide users the kind of information they have to reverse engineer on existing clouds to optimize their applications.

\subsection{Cloud Monitoring Data Collection and Retention}
\subsection{Cloud Monitoring Data Visualization} 

%\subsection{Cloud Monitoring Data Transformation and Visualization}
%During this phase, we will extend the platform we built in the previous phase by filtering and transforming the retained data. Then we will provide an API  for efficient queries and an end-user tool that can visualize the status of the cloud as well as user specific utilization of the cloud.  

%The data collected from the cloud is hard to use and there are some challenges for us to filter and transform the data. First, for monitoring the same cloud, we use different monitoring tools for different levels or even layers, which means that we need a mechanism to relate data from different levels or layers. Second, the data we collected from the system is raw, which means that users may not know what does the data represents and the data need to be aggregated based on users’ demands and use cases. For example, the utilization information come from monitoring tools is compute based, we can aggregate the compute utilizations to get utilizations of a cluster. 

%Having a platform that collects and retains data that represents the status of the cloud, however, is not enough. First, as database schema may change or we may have several databases for storing all the information, direct access to database may not be an efficient way. Second, reading documentations and statistics for large set of data is not yet clear to cloud users, especially for end users. Third, exposing the data of the whole platform means risking the security of the cloud and privacy of users. So we plan to provide an API and an end-user visualization tool for showing cloud status and users’ virtual resources. 
% ************************************************************************
% ****************** Prior Work ****************************************


\section{Prior Work}
\label{sec:PriorWork}

Monitoring is a core function that cloud should support to meter the complex infrastructure, ranging from physical resources to virtualized resources. The monitoring data has been used for capacity and resource planning and management, data center management, SLA management, billing, troubleshooting and performance management~\cite{Aceto2013}. A lot of works has been done to improve granularity, accuracy, scalability, adaptability and elasticity of monitoring solutions in cloud. Rich monitoring solutions [13] [14] [15] have been proposed in the recent years aimed to deal with all above issues. The mature monitoring solutions in the cloud prove the possibility of collecting resources measurements and user behaviors in different layers in the cloud. I have started investigating Sensu, which is a plugin-architected, scalable, adaptive monitoring application for cloud infrastructure.
There have been some literatures mentioned how to collect data for machine learning research. In [6], the authors set up different sets of computers running HPC applications with different numbers of CPU cores, obtained different performance by manual human changes of the number of cores. This method is apparently not scalable and very specific for this research. In [16], the author proposed an idea about optimizing specific application performance by analyzing data of partial execution of the application, which is actually weakening of the power of historical data. [17] realize the major problems for most researches that it is both labor and resource intensive, and does not provide any assurance of the quality of results. So they proposed a simulation tool for predicting application performance in different simulated cloud structures. However, this simulation tools fail providing a real resources usage as the true cloud environment. These works convince us it’s necessary to provide researches community access to information of an actual cloud environment.
OpenStack is an open source project supported by a large community of developers and companies that allows creating and managing large-scale Cloud IaaS deployment. OpenStack provides Compute service for controlling the VM lifecycle and configuration in specific node, and Network service to maintain and manage the virtual networks. Apart from other services, OpenStack provides a metering service called Ceilometer to reliably collect measurements of the utilization of the physical and virtual resources from user perspective, persist these data for subsequent retrieval and analysis [11]. Because of the flexibility of OpenStack, some researches about cloud efficiency have been done on it. More importantly, its metering service provides rich measurement data, so it is the best cloud environment for my project [12].

\section{Proposed Approach}
\label{sec:ProposedApproach}

To achieve our project goals, we will start out our project with a shared part, which is the platform building. In order to expose data to users for my project and to categorize VMs for Hua?s project, we need to collect and retain operational data from the cloud. 

  In order to collect the data, we need to evaluate several monitoring tools in terms of scalability, elasticity, adaptability [cloud monitoring: a survey]. This project will mainly focus on physical system monitoring and usage of virtual resources. From some literatures, we learn that Sensu, Nagios and a lot of monitoring tools can collect the information of cloud��s physical system. Apart from low-level monitoring, we still need some high-level monitoring to provide virtual resources utilization to users. MOC uses OpenStack[4] as its platform to provision VMs and storage units. OpenStack has a generic metering project called ceilometer[18]. We need to install and test ceilometer and compare ceilometer with other metering tools, such as StackTach. Additionally, this project will investigate monitoring of network switches, which are not a common features of most monitoring tools. 
  
  Data collection is another big challenge this project needs to address for platform building. As monitoring event produce huge amount of data, so the storage must be scalable to support efficient queries of massive amount of data in the future. We also need to explore characteristics of distributed databases, as this can increase fault-tolerance, which is desired in a cloud environment. And the data schema needs to be defined in an normalized form amenable to data analysis. Moreover, as we have potential data from logs, monitoring tools, and metering tools, all of them may have its own best fit for storage. Retaining data in a single database or distributing them among different database is one critical question we will study. So far we have two candidates, MongoDB and InfluxDB. MongoDB is a mature distributed NoSQL database while InfluxDB is a time series database. We will evaluate them by looking into literatures and trialing on them in terms of all issues.
  
  After building the platform that is functioned to collect and retain certain amount of data. Hua and I will focus on different parts of the project. I will separate my part of the project to three sections. First, as the project goes along, we will get larger set of data. This part will focus in maintaining the data collected and store it in a way that is well organized and optimized for querying. Storing data that are less useful seem to be a waste of storage and can lower the query speed. So we will do literature reviews, and find out demands of users when they consider using a cloud. Based on users? demand information and users concern such as prices, performance, energy efficiency and time, we can come up with a list of metrics that are useful to collect. For each of the component in the cloud, we need to find parameters that can influence the factors that users are concerned with, such as performance. Furthermore, we need store some useful aggregations of data. People sometimes are more concerned about the usage over a period of time than just instantaneous data. However, monitoring tools usually get instantaneous data only. We will define a schema that can store large set of historical data such as hourly cpu utilization or daily cpu utilization of a compute node.
  
  Second, after retaining all the useful information in one or more database, we need to expose instantaneous cloud operational data  and some historical data as stated above to users via API in a programmable manner. Literature reviews are needed to explore the queries users want to have when they are using the cloud. For example, we will write an API that can provide users a mapping between their virtual resources and cloud physical resources. This particular API will tell them conditions of  the hosts for their VMs and also present them the physical networks of their VMs. 
  
  Third, with the data collected and stored in the database/databases, we will make a data visualization tool. We need to explore how to present the data in a fashionable and readable way. This data visualization tool should show basic descriptions of the cloud, instantaneous usage of the cloud, and the mapping of physical machines and virtual machines for users. We will also explore open source visualization tools, such as Grafana [14] for visualizing and analyzing data within a time range.



\section{Plan and Schedule}
\label{sec:Plan}

\begin{itemize}
\item{[Sep 2015 - Oct 2015]:} Consider many use cases for users of MOC and come up with a list of metrics that are useful for users about MOC. Use existing monitoring tool in MOC, Sensu, to get physical system information of MOC. Explore other monitoring tools that can get physical system information can evaluate them with Sensu. Collect the information as need from the monitoring tool we chose.
\item{[Oct 2015 - Nov 2015]:} Evaluate metering tools, which are used for calculating how much resource a tenant has used. The options now are ceilometer and StackTach. We need to consider the overhead and quantity of the messages. The we will choose a better choice as our metering tool for mappings. Installation and setup of the metering tool that we chose after evaluation. Write a program for mappings for each project using the metering tool we chose after evaluation.
\item{[Nov 2015 - Dec 2015]:} Explore the data that we can get from network switches, and write programs that can show how the physical hosts are connected via networks. 
\item{[Dec 2015 - Jan 2016]:} Evaluate databases and decide if we should use multiple databases or just one database. If we choose one database, we need to come up with a schema for storing all the data. If we choose multiple databases, we need to come up with a way to have a good connection of the data. 
\item{[Jan 2016 - Feb 2016]:} Using the data we collect and stored in the database, we need to make a visualization tool that can visualize both physical system information and a mapping between physical resources and virtual resources for each user. 
\item{[Feb 2016 - Mar 2016]:} Continue making the visualization tool and also come up with use cases for users and design the API for them. 
\item{[Mar 2016 - Apr 2016]:} Write programs that can provide API for users. 
\item{[Apr 2016 - Jun 2016]:} Refine all parts of the project. Write the report.
\end{itemize}


\bibliographystyle{ieeetr}
\bibliography{main}

\end{document}