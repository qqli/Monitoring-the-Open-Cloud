\subsection{Cloud Monitoring Data Collection and Retention}

Cloud computing, as a new industry with increasing number of users, has many intrinsic advantages. With clouds, developers with innovative ideas can try out their new services, without investing in huge amounts of hardware~\cite{Armbrust2009}. It is also claimed that cloud computing gives users the illusion of infinite computing resources available on demand and allows users to pay for the use of computing resources on a short-term basis as needed~\cite{Armbrust2009}. However, there are still some challenges for cloud computing, such as performance invisibility, data transfer bottlenecks, scalability, and cloud information are kept away from users for optimizations.

  Cloud providers do not provide users with any information about the underlying hardware infrastructure of public clouds, thus promising users the illusion of ideal virtual machines. On the contrary, those virtual machines are far from ideal, and they are hardly predictable in terms of their performance. Michael Conley et al. [17] state in their paper that they conduct a series of experiments in order to understand the scaling behavior of virtualized cloud network and storage resources. Far from getting a satisfactory result, they spent 50,000 dollars getting a result that is highly that may not be repeated. If researchers in [17] are provided with sufficient operational data of physical system or other experimental results of clouds, less effort and less cost would be spent for similar series of experiments and more techniques can be applied to cloud research, such as machine learning and data mining.

  Despite of performance invisibility of clouds, cloud providers also hide network bandwidth information from users. As many applications are bandwidth bound, such as a video streaming application, users may have a job finished much longer in a congested network than in a network that is free. But they are unaware of whether their network is congested or idle, making choices only based on feelings. 


  Even though clouds have many challenges and clouds need to be monitored as stated above, most public clouds, which usually run by a single vendor, hide their operational data from users of the clouds. The Open Cloud Exchange (OCX) is envisioned as a public cloud marketplace in which many stakeholders, rather than just a single provider, participate in implementing and operating the cloud [3]. It is also envisioned that cloud?s operational data can be exposed to researchers for their studies. Massachusetts Open Cloud (MOC) is an experiment environment of OCX, many vendors, such as Dell, Lenovo, and Intel, participate in this project. MOC is the first open cloud that users can use, and potential researchers can do research with resourceful information. As MOC intend to expose data to researchers as well as users, cloud monitoring is also a must in MOC. In addition, with vendors providing many services, MOC is a place that can give users more choices, including brands, performance, price, energy efficiency and so on. Those choices are only realizable if we provide a monitoring solution to MOC. 

  In addition to the monitoring motivation that is triggered by the opacity of current public clouds, cloud providers must monitor the cloud in order to guarantee SLAs. Service provisioning in the Cloud is based on Service Level Agreements(SLAs), which states the term of the service including the quality of service (QoS), obligations, service pricing, and penalties in case of agreement violations [16]. However, cloud environment can be modeled in seven layers: facility, network, hardware, OS, middleware, application, and the user[15]. Traditional monitoring tools cannot be applied to cloud environments, which stress the need of cloud monitoring.

  Cloud monitoring can be divided into high-level monitoring and low-level monitoring[15]. Most of the clouds that provide monitoring services only give high-level monitored information, which is the virtual system of the cloud. Low-level monitoring, which can be considered as the monitoring of clouds? physical system, is often not provided to consumers. 

  Based on the needs of cloud monitoring and the opacity of clouds that hinders user experience and research capabilities, we propose that exposing operational data of the cloud in a readable, organized and meaningful way to help users in making decisions and doing scientific researches. However, there are still many challenges in terms of cloud monitoring. First, how to collect operational data of the cloud from different sources in a scalable way, without impacting the performance of the components being monitored. Second, how to retain the data in an organized way and to optimize the retention for data analysis. Third, how to make use of that data in various research projects. Fourth, how to expose the information in a way that is clear to users and also protect the privacy of the cloud and users. 


  This project and another masters project -'Hua's project name here' will take the first steps towards the challeges current clouds have. We will spend two months building up the platform for data collection and retention. We will mainly focus in low-level monitoring and will come up with a basic set of data to be collected and instrument the MOC to collect that information. Afterwards, this project will continuing in completing the platform by providing users a visualization tool and also an API for access of the data. The other project will focus in the data analysis as a researcher.