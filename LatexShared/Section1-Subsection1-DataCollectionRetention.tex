\subsection{Cloud Monitoring Data Collection and Retention}

One of the main arguments for utilizing public clouds is it is an environment where developers with innovative ideas can try out their new services without investing in huge amounts of hardware~\cite{Armbrust2009}. However, public cloud providers refrain from exposing the performance characteristics and utilization of the cloud software and hardware components from cloud users and researchers. As a result, developers are often forced to do reverse engineering to discover the characteristics of the cloud and optimize their applications. A good example to this is presented in~\cite{???}, where Conley et al. try to identify the best performing instance type for sorting 100TB of data on Amazon EC2. After evaluating 48 different instance types available, and spending \$50.000 along the way, they identified the best performing instance type. That instance type can sort 100TB of data at a cost of \$300. 

Unfortunately, even the authors of~\cite{???} are not exactly sure that their experiments are repeatable as the utilization of the cloud and the resources they used when they performed these experiments are unknown to them. Our goal in this project is to prevent this kind of performance uncertainty by providing the underlying infrastructure utilization information to users so that they can do intelligent decisions.          

However, there are many valid reasons why cloud providers do not want to provide visibility into the performance and utilization characteristics of their cloud infrastructure. First, they want to keep their solutions private due to commercial concerns. Second, sharing details of the underlying cloud infrastructure may lead to security problems. Third, providing information about a user?s resources may expose information about other users in the cloud and cause privacy issues. 

On the other hand, since we envision to implement our monitoring solution on an open cloud setting (MOC), we are not bound by commercial concerns. Also, we believe that security through obscurity is not the right solution for achieving high level cloud security and providing increased transparency to the cloud infrastructure will foster building of more secure cloud solutions. Finally, one of our goals in this project is to come up with ways of exposing cloud utilization information in a manner that preserves user privacy.  

  Even though clouds have many challenges and clouds need to be monitored as stated above, most public clouds, which usually run by a single vendor, hide their operational data from users of the clouds. The Open Cloud Exchange (OCX) is envisioned as a public cloud marketplace in which many stakeholders, rather than just a single provider, participate in implementing and operating the cloud [3]. It is also envisioned that cloud?s operational data can be exposed to researchers for their studies. Massachusetts Open Cloud (MOC) is an experiment environment of OCX, many vendors, such as Dell, Lenovo, and Intel, participate in this project. MOC is the first open cloud that users can use, and potential researchers can do research with resourceful information. As MOC intend to expose data to researchers as well as users, cloud monitoring is also a must in MOC. In addition, with vendors providing many services, MOC is a place that can give users more choices, including brands, performance, price, energy efficiency and so on. Those choices are only realizable if we provide a monitoring solution to MOC. 

  In addition to the monitoring motivation that is triggered by the opacity of current public clouds, cloud providers must monitor the cloud in order to guarantee SLAs. Service provisioning in the Cloud is based on Service Level Agreements(SLAs), which states the term of the service including the quality of service (QoS), obligations, service pricing, and penalties in case of agreement violations [16]. However, cloud environment can be modeled in seven layers: facility, network, hardware, OS, middleware, application, and the user[15]. Traditional monitoring tools cannot be applied to cloud environments, which stress the need of cloud monitoring.

  Cloud monitoring can be divided into high-level monitoring and low-level monitoring[15]. Most of the clouds that provide monitoring services only give high-level monitored information, which is the virtual system of the cloud. Low-level monitoring, which can be considered as the monitoring of clouds? physical system, is often not provided to consumers. 

  Based on the needs of cloud monitoring and the opacity of clouds that hinders user experience and research capabilities, we propose that exposing operational data of the cloud in a readable, organized and meaningful way to help users in making decisions and doing scientific researches. However, there are still many challenges in terms of cloud monitoring. First, how to collect operational data of the cloud from different sources in a scalable way, without impacting the performance of the components being monitored. Second, how to retain the data in an organized way and to optimize the retention for data analysis. Third, how to make use of that data in various research projects. Fourth, how to expose the information in a way that is clear to users and also protect the privacy of the cloud and users. 


  This project and another masters project -'Hua's project name here' will take the first steps towards the challeges current clouds have. We will spend two months building up the platform for data collection and retention. We will mainly focus in low-level monitoring and will come up with a basic set of data to be collected and instrument the MOC to collect that information. Afterwards, this project will continuing in completing the platform by providing users a visualization tool and also an API for access of the data. The other project will focus in the data analysis as a researcher.