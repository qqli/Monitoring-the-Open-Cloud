\subsection{Cloud Monitoring Data Collection and Retention}

To achieve our project goals, we will start out our project with a shared part, which is the platform building. In order to expose data to users for my project and to categorize VMs for Hua?s project, we need to collect and retain operational data from the cloud. 

  In order to collect the data, we need to evaluate several monitoring tools in terms of scalability, elasticity, adaptability~\cite{Aceto2013}. This project will mainly focus on physical system monitoring and usage of virtual resources. From some literatures, we learn that Sensu, Nagios and a lot of monitoring tools can collect the information of cloud's physical system. Apart from low-level monitoring, we still need some high-level monitoring to provide virtual resources utilization to users. MOC uses OpenStack[4] as its platform to provision VMs and storage units. OpenStack has a generic metering project called ceilometer~\cite{ceilometer}. We need to install and test ceilometer and compare ceilometer with other metering tools, such as StackTach. Additionally, this project will investigate monitoring of network switches, which are not a common features of most monitoring tools. 
  
  Data collection is another big challenge this project needs to address for platform building. As monitoring event produce huge amount of data, so the storage must be scalable to support efficient queries of massive amount of data in the future. We also need to explore characteristics of distributed databases, as this can increase fault-tolerance, which is desired in a cloud environment. And the data schema needs to be defined in an normalized form amenable to data analysis. Moreover, as we have potential data from logs, monitoring tools, and metering tools, all of them may have its own best fit for storage. Retaining data in a single database or distributing them among different database is one critical question we will study. So far we have two candidates, MongoDB and InfluxDB. MongoDB is a mature distributed NoSQL database while InfluxDB is a time series database. We will evaluate them by looking into literatures and trialing on them in terms of all issues.