\subsection{Cloud Monitoring Data Collection and Retention}

For commercial platforms, some of the major companies such as Amazon and Microsoft, already provide monitoring services. CloudWatch, a monitoring service provided by Amazon, only have services of monitoring virtual resources such as EC2~\cite{Aceto2013}. Amazon does not provide low-level information (physical machines€™ information). AzureWatch, provided by Microsoft, also does not provide the monitoring information of low-level cloud. 

  For open source platforms, there are some monitoring tools focusing on different aspects of cloud computing infrastructures. As I focus on monitoring physical system of the cloud, I explored some monitoring tools that monitor physical systems. Nagios [6] is a widely-used enterprise-level open source monitoring tool that can monitor cloud'€™s physical machines. Sandoval et cl.[7] analyze several already available monitoring tools, such as Nagios, HypericHQ, Lattice, Zenoss, and indicate that Nagios is the best choice. 

  Sensu[8] is designed as a publish/subscribe based monitoring tool that can deal with some of the problems that Nagios cannot solve. In Nagios, there is a configuration file that needs to be modified and restart Nagios whenever we want to have any modification. However, sensu-clients, which run on remote hosts, can subscribe to some checks and can avoid restarting the whole monitoring tool. It also uses RabbitMQ, which is a message-oriented middleware that implements Advanced Message Queueing Protocol (AMQP). Sensu is said to be a monitoring tool that is scalable, extensible and elastic~\cite{Aceto2013}. 